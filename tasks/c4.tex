\documentclass[a4paper,11pt]{article}
\usepackage[margin=2cm,bottom=2cm]{geometry}
\usepackage{amsmath,amsthm,amsopn,amsfonts}
\usepackage{polski}
\usepackage[utf8]{inputenc}
\usepackage{enumerate}
\usepackage{graphicx}
\pagestyle{empty}

\newcommand{\makeheader}[1]{
\begin{center}
{\Large\bf\sectfont Algorytmy online\\}
\vspace{0.3cm}
{\large\bf\sectfont Lista #1}\\
\end{center}
\vspace{0.7cm}
}


\newcommand{\myfigure}[3]{
\begin{figure}[t]
\centering
\fboxsep7pt
\framebox[0.95\textwidth]{
    \begin{minipage}{0.95\textwidth}
        \centering #3
    \end{minipage}
}
\caption{#2}
\label{fig:#1}
\end{figure}
}



\newcommand{\makefooter}{
\bigskip
\hfill {\em Marcin Bieńkowski}
}

\newcommand{\E}{\mathbf{E}}
\newcommand{\ALG}{\textsc{Alg}}
\newcommand{\OPT}{\textsc{Opt}}
\newcommand{\e}{\textnormal{e}}
\newcommand{\CNT}{\textsc{Cnt}}


\begin{document}
\makeheader{4}


\begin{description}

%\item[Zadanie 1] \textbf{(4 pkt.)} 
%Na wykładzie zaprezentowany został algorytm \textsc{Rand} dla problemu pamięci
%podręcznej, który --- w przypadku chybienia --- wyrzuca z pamięci podręcznej
%losową stronę. Pokaż, że \textsc{Rand} jest $k$-konkurencyjny. (To powinien być pełen dowód, 
%tzn.~trzeba przeanalizować również przypadki rozważone już na wykładzie).

\item[Zadanie 1.] \textbf{(2 pkt.)} 
    Rozważmy deterministyczny algorytm \textsc{Half-MTF} dla statycznej reorganizacji listy, 
    który obsługuje odwołanie do elementu $x$ w następujący sposób:
    jeśli jest to parzyste odwołanie do elementu $x$, to $x$ zostaje przeniesiony na początek.
	Pokaż, że \textsc{Half-MTF} jest $3$-konkurencyjny.    

% Analizujemy bita w najgorszym mozliwym (nie-srednim) przypadku

\item[Zadanie 2.] \textbf{(4 pkt.)} 
    Rozważ algorytm \textsc{Rand-MTF} dla problemu reorganizacji listy o długości $\ell$, który
    w przypadku odwołania do elementu $x$ przesuwa go na początek listy z prawdopodobieństwem $1/2$. 
    Udowodnij, że współczynnik konkurencyjności tego algorytmu wynosi co najmniej $2 - O(1/\sqrt{\ell})$.
	\footnote{Można zastąpić $O(1/\sqrt{\ell})$ inną funkcją malejącą wraz z $\ell$.}

    Wskazówka: Niech początkowa lista zawiera elementy $x_1, x_2, \dots, x_\ell$. 
    Załóż, że współczynnik wynosi $2 - \epsilon$ i rozważ 
    ciąg odwołań do elementów $x_\ell^k, x_{\ell-1}^k, \ldots, x_1^k$ powtórzony wiele razy.

% Zawsze jest dobrze = podczas tych k krokow przenosimy dany element na poczatek. 
% E[ALG] = E[ALG|zawsze dobrze] * Pr[zawsze dobrze] + E[ALG|nie zawsze dobrze] * Pr[nie zawsze dobrze]
% drugi skladnik jest wiekszy od zera. Pr[zawsze dobrze] moze byc dowolnie bliskie 1 jesli zwiekszymy k
% Z kolei E[ALG|zawsze dobrze] moze byc dowolnie bliskie 2 * OPT.

\item[Zadanie 3.] \textbf{(4 pkt.)} 
Na wykładzie zaprezentowany został algorytm \textsc{Rand} dla problemu pamięci
podręcznej, który --- w przypadku chybienia --- wyrzuca z pamięci podręcznej
losową stronę. Pokaż, że konkurencyjność takiego algorytmu wynosi co najmniej
$k$.\footnote{Na wykładzie wspomniałem, że $k$-konkurencyjność algorytmu 
\textsc{Rand} zachodzi również dla silniejszego, adaptującego się adwersarza. Należy
rozwiązywać to zadanie, jak gdyby tej uwagi nie było, tj.~pokazać dolne
ograniczenie wykorzystując słabszego adwersarza, który nie widzi bitów losowych
algorytmu (czyli takiego, jak rozważaliśmy do tej pory).}

\end{description}

\makefooter
\end{document}

