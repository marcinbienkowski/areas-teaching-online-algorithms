\documentclass[a4paper,11pt]{article}
\usepackage[margin=2cm,bottom=2cm]{geometry}
\usepackage{amsmath,amsthm,amsopn,amsfonts}
\usepackage{polski}
\usepackage[utf8]{inputenc}
\usepackage{enumerate}
\usepackage{graphicx}
\pagestyle{empty}

\newcommand{\makeheader}[1]{
\begin{center}
{\Large\bf\sectfont Algorytmy online\\}
\vspace{0.3cm}
{\large\bf\sectfont Lista #1}\\
\end{center}
\vspace{0.7cm}
}


\newcommand{\myfigure}[3]{
\begin{figure}[t]
\centering
\fboxsep7pt
\framebox[0.95\textwidth]{
    \begin{minipage}{0.95\textwidth}
        \centering #3
    \end{minipage}
}
\caption{#2}
\label{fig:#1}
\end{figure}
}



\newcommand{\makefooter}{
\bigskip
\hfill {\em Marcin Bieńkowski}
}

\newcommand{\E}{\mathbf{E}}
\newcommand{\ALG}{\textsc{Alg}}
\newcommand{\OPT}{\textsc{Opt}}
\newcommand{\e}{\textnormal{e}}
\newcommand{\CNT}{\textsc{Cnt}}


\begin{document}
\makeheader{5}


\begin{description}

\item[Zadanie 1.] \textbf{(4 pkt.)} 
	Rozważmy wariant problemu MTS ({\em Metrical Task System}) z metryką dyskretną 
	\mbox{$n$-elementową}, gdzie w każdym kroku wektor kar może być dowolny.
	Pokaż dolne ograniczenie $\Omega(\log n)$ na konkurencyjność dowolnego algorytmu 
	randomizowanego.

\item[Zadanie 2.] \textbf{(3 pkt.)} 
    Pokaż, że współczynnik ścisłej konkurencyjności dowolnego 
    randomizowanego algorytmu dla problemu znajdowania krowy 
    wynosi co najmniej~$2.01$.
    % krowa może być w punktach -2, -1, 1, 2 w kazdym z takim samym ppb => lower bound 2.333

\item[Zadanie 3.] \textbf{(3 pkt.)} 
	Pokaż dolne ograniczenie na konkurencyjność dowolnego randomizowanego 
    algorytmu rozwiązującego problem wypożyczania nart wynoszące $\e/(\e-1)$, gdy $B \to \infty$.
    \emph{Wskazówka: zastosuj zasadę minimaksową Yao stosując następujący rozkład:
    narciarz łamie nogę dnia $1 \leq i \leq B$ z prawdopodobieństwem 
    \begin{align*}
    p_i & = \frac{1}{B} \cdot \left( \frac{B-1}{B}\right)^{i-1},
    \intertext{i nigdy nie łamie nogi z prawdopodobieństwem}
    p_\infty & = \left( \frac{B-1}{B} \right)^B.
    \end{align*}}
    % E[ALG] = B, E[OPT] = B(1-((b-1)/b)^b) 

\end{description}

\makefooter
\end{document}

