\documentclass[a4paper,11pt]{scrartcl}
\usepackage[margin=2cm,bottom=2cm]{geometry}
\usepackage{amsmath,amsthm,amsopn,amsfonts}
\usepackage{polski}
\usepackage[utf8]{inputenc}
\usepackage{enumerate}
\usepackage{graphicx}
\pagestyle{empty}

\newcommand{\makeheader}[1]{
\begin{center}
{\Large\bf\sectfont Algorytmy online\\}
\vspace{0.3cm}
{\large\bf\sectfont Lista #1}\\
\end{center}
\vspace{0.7cm}
}


\newcommand{\myfigure}[3]{
\begin{figure}[t]
\centering
\fboxsep7pt
\framebox[0.95\textwidth]{
    \begin{minipage}{0.95\textwidth}
        \centering #3
    \end{minipage}
}
\caption{#2}
\label{fig:#1}
\end{figure}
}



\newcommand{\makefooter}{
\bigskip
\hfill {\em Marcin Bieńkowski}
}

\newcommand{\E}{\mathbf{E}}
\newcommand{\ALG}{\textsc{Alg}}
\newcommand{\OPT}{\textsc{Opt}}
\newcommand{\e}{\textnormal{e}}
\newcommand{\CNT}{\textsc{Cnt}}


\begin{document}
\makeheader{7}

\begin{description}

\item[Zadanie 1] \textbf{(4 pkt.)}
Rozważmy wariant ułamkowy problemu kupowania nart: w każdym dniu algorytm 
wybiera jaki ułamek nart $x \in [0,1]$ chce posiadać. Wartość $x$ można tylko 
zwiększać, tj.~nie można nart (częściowo) odsprzedawać. 
Następujący program liniowy $(\mathcal{P}_k)$ opisuje 
ten problem po $k$ dniach, gdzie $z_j$ jest ułamkiem nart, który pożyczymy dnia $j$.
\begin{align*}
\textnormal{minimalizuj: } & B \cdot x + \sum_{j=1}^k z_j & \\
\textnormal{przy warunkach: }
    & x + z_j \geq 1 & \textnormal{dla każdego dnia $j \leq k$} \\
    & x \geq 0, \enspace z_j \geq 0 &  
\end{align*}
Zauważmy, że program dualny wygląda następująco $(\mathcal{D}_k)$:
\begin{align*}
\textnormal{maksymalizuj: } & \sum_{j=1}^k y_j & \\
\textnormal{przy warunkach: }
	& \sum_{j=1}^k y_j \leq B \\
    & 0 \leq y_j \leq 1 & \textnormal{dla każdego dnia $j \leq k$} 
\end{align*}

Rozważmy następujący algorytm dla tego problemu. Początkowo ustalamy $x = 0$. 
W dniu $k$, jeśli $x \geq 1$, wykonujemy przypisania 
\begin{align*} 
z_k \leftarrow 0 \enspace, 
\hspace{0.5cm} 
y_k \leftarrow 0 \enspace,
\end{align*}
zaś w przeciwnym przypadku ($x < 1$) przypisania
\begin{align*} 
z_k \leftarrow 1-x \enspace,
\hspace{0.5cm} 
x \leftarrow (1+1/B) \cdot x + c \enspace,
\hspace{0.5cm} 
y_k \leftarrow 1 \enspace.
\end{align*}
gdzie $c$ jest pewną stałą. Jakie może być $c$, żeby generowane rozwiązania dla 
$(\mathcal{P}_k)$ i $(\mathcal{D}_k)$ były dopuszczalne? Jak współczynnik (ścisłej) konkurencyjności 
rozwiązania zależy od $c$? Pokaż, że można tak dobrać $c$, żeby współczynnik był równy
\[ 
	R = \frac{(1+1/B)^B}{(1+1/B)^B-1} 
	\enspace.
\]

\item[Zadanie 2] \textbf{(2 pkt.)}
Rozważmy następujące losowe zaokrąglenie podejścia opisanego w poprzednim
zadaniu. Na początku algorytm wybiera losowo liczbę $\alpha \in [0,1]$.
Następnie symuluje algorytm ułamkowy i w kroku, w którym nowo obliczona wartość
$x$ wynosi co najmniej $\alpha$, kupuje narty. Pokaż, że konkurencyjność takiego
algorytmu jest nie większa niż konkurencyjność rozwiązania ułamkowego.

\item[Zadanie 3] \textbf{(2 pkt.)}
Rozważ algorytm dla ułamkowego wariantu problemu pokrywania zbiorami, który w każdym kroku aktualizuje
obecne rozwiązanie zachłannie minimalizując przyrost kosztu. Pokaż dolne ograniczenie $\Omega(n)$
na ścisłą konkurencyjność takiego algorytmu, gdzie $n$ jest liczbą elementów uniwersum.
% Jeden zbior ktory ma wszystkie elementy o koszcie 1+eps, wszystkie inne zbiory sa singletonami 
% o kosztach = 1. Algorytm zawsze bierze te singletony.


\item[Zadanie 4] \textbf{(2 pkt.)}
Rozważmy program prymalny i dualny (z wykładu) dla ułamkowego wariantu problemu pokrywania zbiorami. 
Nie korzystając ze słabego twierdzenia o dualności pokaż, że wartość dowolnego rozwiązania 
dla programu prymalnego jest co najmniej taka jak wartość dowolnego rozwiązania programu 
dualnego.\footnote{Rozwiązaniem tego zadania jest dowód słabego twierdzenia o dualności, 
ale wykorzystujący konkretne zmienne i nierówności z problemu pokrywania zbiorami.}


\end{description}

\makefooter
\end{document}

