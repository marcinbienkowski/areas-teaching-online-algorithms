\section{Wstęp do algorytmów online}

\marginnote{W.1A\\+10m}

\problem{Problem optymalizacyjny}{
\begin{compactitem}
\item Zbiór wejść $\I$.
\item Dla każdego wejścia $\sigma \in \I$: zbiór {\em dopuszczalnych} rozwiązań $\F(\sigma)$.
\item Funkcja kosztu $\COST(\sigma,s) \in \REAL_{\geq 0}$ (gdzie $s \in \F(\sigma)$).
\item $\min$ lub $\max$.
\end{compactitem}
}

Będziemy rozważać głównie {\em problemy minimalizacyjne}.
Dla algorytmu {\ALG} przez $\ALG(\sigma)$ czasem będziemy rozumieć koszt 
rozwiązania generowanego przez {\ALG} na $\sigma$ a czasem samo rozwiązanie. 

\begin{definition}{Rozwiązanie optymalne:}
\[
	\OPT(\sigma) = \min_{s \in \F(\sigma)} \COST(\sigma,s)
\]
\end{definition}

\begin{definition}
Deterministyczny {\ALG} jest $R$-aproksymacyjny jeśli dla każdego $\sigma \in \I$
zachodzi 
\[ \ALG(\sigma) \leq R \cdot \OPT(\sigma) \enspace. \]
\end{definition}

\noindent
\textbf{Algorytm online:} algorytm, który dostaje wejście ,,po kawałku'' i ,,po kawałku'' musi 
wypisywać wyjście. 

\begin{definition}
Deterministyczny {\ALG} jest {\em ściśle $R$-konkurencyjny}, 
jeśli jest algorytmem online i jest $R$-aproksymacyjny.
\end{definition}

\noindent
\textbf{Uwagi:}
\begin{compactitem}
\item Wciąż porównujemy nasz algorytm {\ALG} z najlepszym możliwym rozwiązaniem {\em offline}
{\OPT}.
\item Nie walczymy zazwyczaj z niewystarczającą mocą obliczeniową (jak w aproksymacji) tylko
i wyłącznie z nieznajomością przyszłości.
\end{compactitem}


\subsection{Problemy typu wypożycz-lub-kup}

%%%%%%%%%%%%%%%%%%%%%%%%%%%%%%%%%%%%%%%%%%%%%%%%%%%%%%%%%%%%%%%%%%%%%%%%%%%%%%%%%

\problem{Problem wypożyczania nart (Ski Rental Problem, SRP)}{Codziennie rano podejmujemy decyzję, czy
narty pożyczyć (za $1$) czy też kupić (za $B \in \NAT$). Jeździmy przez cały
dzień a wieczorem łamiemy nogę (i kończymy swoją narciarską karierę) lub też
nie. Cel: minimalizacja całkowitego kosztu.}

\noindent
Niech $T \in \mathbb{N}_{+} \cup \{ \infty \}$ oznacza wybrany przez adwersarza dzień, w którym łamana jest noga 

\begin{observation}
Rozwiązanie optymalne to kupno nart pierwszego dnia jeśli $B < T$, a wypożyczanie nart 
każdego dnia w przeciwnym przypadku. Zatem $\OPT(\sigma) = \min\{B,T\}$.
\end{observation}

\noindent
Jak konkurencyjna jest strategia ,,kup pierwszego dnia''? A strategia ,,zawsze pożyczaj''?

\begin{theorem}
\label{thm:srp-2-competitive}
Niech {\ALG} pożycza narty przez $B-1$ dni, a dnia $B$ je kupi
(pod warunkiem, że wcześniej nie złamie nogi).
Wtedy $\ALG$ jest $(2 - 1/B)$-konkurencyjny.
\end{theorem}

\begin{proof}
Weźmy dowolną sekwencję wejściową $\sigma$. Niech $T$ będzie liczbą dni w których
narciarz ma jeszcze sprawne nogi (tzn.~noga jest łamana pod koniec dnia $T$).
\[
\ALG(\sigma) = \begin{cases}
  T & \textnormal{jeśli $T \leq B - 1$} \\
  B - 1 + B & \textnormal{jeśli $T \geq B$} 
\end{cases} \enspace.
\]
Porównując to z $\OPT(\sigma) = \min\{B,T\}$ otrzymujemy, że 
dla dowolnego wejścia $\sigma$
\[
  \ALG(\sigma) \leq \left(2 - \frac{1}{B}\right) \cdot \OPT(\sigma) \enspace. \qedhere
\]
\end{proof}

\begin{theorem}
Ścisła konkurencyjność dowolnego algorytmu deterministycznego dla problemu wypożyczania nart 
wynosi co najmniej $(2 - \frac{1}{B})$.
\end{theorem}

\begin{proof}
WLOG, {\ALG} jest zdefiniowany jako ,,kup narty dnia $i$'', 
gdzie $i \in \NAT \cup \{ \infty \}$.
Co robi adwersarz? Pozwala algorytmowi kupić narty dnia $i < B$ i w dniu $i$
łamie mu nogę. Wtedy:
\[
  \frac{\ALG(I)}{\OPT(I)} \;=\; \frac{i-1+B}{\min\{i,B\}} \;\geq \; 2 - \frac{1}{B} 
  \enspace .
  \qedhere
\]
\end{proof}

%%%%%%%%%%%%%%%%%%%%%%%%%%%%%%%%%%%%%%

\problem{Problem spin-block}{
Proces chce uzyskać dostęp do zasobu, który jest aktualnie zajęty. Jądro systemu 
może najpierw czekać dowolny czas ({\em spin}) na zwolnienie zasobu, a następnie 
--- jeśli zasób nie zostanie zwolniony --- zatrzymać proces ({\em block}) i przekazać sterowanie 
do innego procesu, co trwa $B$ ms. Celem jest minimalizacja czasu przeznaczonego na uzyskanie dostępu do zasobu.
}

Decyzja algorytmu = ile należy czekać? 
Deterministyczny algorytm online osiągający optymalny współczynnik konkurencyjności $2$ 
jest analogiczny dla SRP: czeka $B$ ms na zwolnienie zasobu.
Można go poprawić stosując randomizację.

\begin{definition}
Randomizowany algorytm online jest $R$-konkurencyjny, jeśli dla każdego wejścia 
$\sigma$ zachodzi
\begin{align}
	\label{eq:konkurencyjnosc_rand}
	\E[\ALG(\sigma)] & \leq \R \cdot \OPT(\sigma) 
	\enspace,
\end{align}
gdzie wartość oczekiwana jest liczona po wszystkich wyborach losowych algorytmu.
\end{definition}





Poniżej pokażemy, że współczynnik ten można poprawić stosując randomizację.
Niech $\RAND$ będzie algorytmem, który losuje (z rozkładem jednostajnym) liczbę $x$ z przedziału $[B/2,B]$
i czeka $x$ ms na zwolnienie zasobu.

\begin{theorem}
Algorytm $\RAND$ jest $1,75$-konkurencyjny
\end{theorem}

\begin{proof}
Weźmy dowolne wejście $I$; niech $T$ będzie momentem w którym zasób zostaje zwolniony. 
Oczywiście $\OPT(I) = \min\{T,B\}$. 
Rozpatrzmy trzy przypadki. 
\begin{enumerate}
\item $T > B$. Ponieważ zarówno $\OPT$ jak i $\RAND$ blokują zawsze przed czasem $B$, ich koszt nie zmieni się
	jeśli założymy, że $T = B$ (przypadek rozpatrzony w kolejnym punkcie).
\item $T < B/2$. W takim przypadku, $\RAND(I) = T = \OPT(I)$. 
\item $B/2 \leq T \leq B$. Wtedy $\OPT(I) = T$.
Zauważmy, że gęstość rozkładu prawdopodobieństwa w przedziale $[B/2,B]$ to $2/B$.
Niech $\RAND(I,x)$ oznacza koszt algorytmu $\RAND$ na wejściu $I$ pod warunkiem, że $\RAND$ wylosował $x$. 
Wtedy oczekiwany koszt algorytmu wynosi 
\begin{align*}
	\E[\RAND(\sigma)] 
		= &\; \int_{B/2}^B \RAND(I,x) \cdot \frac{2}{B} \, dx 
		= \frac{2}{B} \cdot \left(\int_{B/2}^T \RAND(I,x) \, dx +
			  \int_T^B \RAND(I,x) \, dx \right)\\
		= &\; \frac{2}{B} \cdot \left( \int_{B/2}^T (x+B) \, dx +
			  \int_T^B T \, dx \right) 
		=  \frac{2}{B} \cdot \left(\frac{T^2}{2}+B \cdot T - \frac{(B/2)^2}{2} - \frac{B^2}{2}
			 + B \cdot T - T \cdot T \right)  \\
		= &\; \frac{2}{B} \cdot \left(2 \cdot B \cdot T - \frac{5}{8} \cdot B^2 - T^2/2 \right)
 		= 4 \cdot T - \frac{5}{4} \cdot B - T^2/B 
		\enspace.
\end{align*}
Stąd wynika, że $\E[\RAND(I)] / \OPT(I) = 4 - \frac{5}{4} \cdot B/T - T/B$.
Obliczając pochodną, otrzymujemy, że wyrażenie po prawej stronie maksymalizowane jest 
dla $T = \frac{\sqrt{5}}{2} \cdot B$. Jednak taka wartość jest poza przedziałem $[B/2,B]$ i dlatego  współczynnik jest największy na jednym z~końców (prawym) przedziału. Zatem
\[
	\frac{\E[\RAND(I)]}{\OPT(I)} \leq 4 - \frac{5}{4} \cdot B/B - B/B = 1,75
	\enspace.
	\qedhere
\]
\end{enumerate}
\end{proof}

%%%%%%%%%%%%%%%%%%%%%%%%%%%%%%%%%%%%%%

\subsection{Eksploracja terenu}
\marginnote{W.1B}


\problem{Wejście na pastwisko}{
Krowa stoi  przy drodze w pewnej odległości do wejścia na pastwisko (większej od~$1$). 
Krowa jest ślepa i chce po omacku znaleźć wejście przechodząc jak najmniejszą odległość.
}

\begin{algorithmic}
\State $d \gets 1$
\State $strona \gets lewo$ 
\Repeat
	\State Idź $d$ kroków w stronę $strona$ 
	\State Wróć do punktu startowego 
	\State $d \gets d \cdot 2$ 
	\State $strona \gets lewo$ 
\Until{kwejście znalezione}
\end{algorithmic}

\begin{theorem}
Powyższy algorytm jest ściśle $9$-konkurencyjny.
\end{theorem}

\begin{proof}
Weźmy dowolną instancję wejściową $\sigma$. 
Niech $x$ oznacza odległość wejścia od krowy.
Oczywiście, $\OPT(\sigma) = x$. Skoro $x > 1$ to istnieje liczba naturalna 
$j$ spełniająca zależność 
\[ 2^j < x \leq 2^{j+1} \enspace. \]
W najgorszym przypadku krowa posługująca się powyższym algorytmem w iteracji $j$ przejdzie 
$2^j$ kroków w stronę wejścia, zawróci do punktu startowego, przejdzie $2^{j+1}$ w przeciwnym kierunku,
zawróci do punktu startowego i w końcu przejdzie $x$ kroków napotykając wejście na pastwisko.
Całkowity koszt algorytmu wyniesie:
\begin{align*} 
\ALG(\sigma) \;& =\; 2 \cdot 1 + 2 \cdot 2 + 2 \cdot 4 + \ldots 2 \cdot 2^j + 2 \cdot 2^{j+1} + x \\
	& = \;2 \cdot (1 + 2 + \ldots + 2^{j+1}) + x\\
	& \leq\; 2^{j+3} + x 
	\leq\; 8 \cdot 2^j + x 
	<\; 9 \cdot x 
	=\; 9 \cdot \OPT(\sigma) \enspace.
\qedhere
\end{align*}
\end{proof}


%%%%%%%%%%%%%%%%%%%%%%%%%%%%%%%%%%%%%%

\subsection{Pakowanie pojemników}

\problem{Pakowanie pojemników}{
Mamy nieskończony zbiór pojemników o rozmiarach równych $1$. Ciąg
wejściowy składa się z listy przedmiotów o wagach $w_i \leq 1$. 
Algorytm musi decydować, gdzie umieścić dany przedmiot, tak żeby 
suma wag w każdym pojemniku była co najwyżej $1$.
Celem jest minimalizacja liczby zużytych pojemników.
}

Algorytm $\FF$ ($\FFS$) wkłada przedmiot do pierwszego pojemnika, do którego 
włożenie jest możliwe.

\begin{lemma}
Dla dowolnego wejścia $\sigma$ zachodzi $\FF(\sigma) \leq 2 \cdot \OPT(\sigma) + 1$.
\end{lemma}


\begin{proof}
Weźmy dowolny ciąg wejściowy $\sigma$ zawierający $m$ przedmiotów.
Niech $B$ oznacza zbiór wykorzystanych przez $\FF$ pojemników. Pokażemy, że we wszystkich 
pojemnikach, być może poza jednym, sumaryczna waga przedmiotów wynosi co najmniej $1/2$.
Załóżmy nie wprost, że są dwa takie pojemniki $b_i$ i $b_j$ (gdzie $i \leq j$) 
i rozważmy sytuację w momencie, w którym wkładany był pierwszy przedmiot do pojemnika $b_j$. 
Przedmiot ten miał wagę co najwyżej $1/2$ a dodatkowo w pojemniku $b_i$ było wtedy co najmniej 
$1/2$ miejsca. Zatem przedmiot ten nie powinien był zostać włożony do ,,późniejszego'' pojemnika $b_j$.

Niech $k$ oznacza liczbę pojemników wykorzystanych przez algorytm $\FF$. Bez straty ogólności
załóżmy, że wszystkie pojemniki poza ostatnim mają zgromadzoną wagę przynajmniej $1/2$. Zatem
\[
	\FFS(\sigma) \;=\; k \;=\; 1 + (k-1) 
		\;\leq\; 1 + \sum_{i = 1}^{k-1} 2 \cdot w(b_i) 
		\;\leq\; 1 + 2 \cdot \sum_{j=1}^m w(\sigma_j) 
		\;\leq\; 1 + 2 \cdot \OPT(\sigma) \enspace.
		\qedhere
\]
\end{proof}

Powyższy wynik stanowi motywację do wprowadzenia konkurencyjności (nie-ścisłej).

\begin{definition}
Deterministyczny algorytm $\ALG$ jest $\R$-konkurencyjny jeśli istnieje stała $\alpha$, taka że 
dla dowolnego wejścia $I$ zachodzi
\begin{equation}
	\ALG(I) \leq \R \cdot \OPT(I) + \alpha
	\enspace. 
\end{equation}
W analogiczny sposób modyfikujemy definicję dla algorytmów zrandomizowanych. 
\end{definition}

\begin{corollary}
$\FF$ jest $2$-konkurencyjny.
\end{corollary}

