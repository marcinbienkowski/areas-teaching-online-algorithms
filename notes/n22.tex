
%%%%%%%%%%%%%%%%%%%%%%%%%%%%%%%%%%%%%%


\section{K-Server: Algorytm Balance}

W tej sekcji pokażemy, że algorytm \textsc{Balance} jest $k$-konkurencyjny dla dowolnych $k+1$-punktowych 
przestrzeni metrycznych. Definiujemy $D(v_i)$ jako długość drogi, którą przebył serwer obecnie znajdujący się 
w wierzchołku $v_i$ (jest tylko jeden taki serwer). 
Najpierw pokażemy następujący lemat.

\vspace{0.3cm}
\begin{lemma}
\label{lem:balance}
Dla dowolnych $v_i$ i $v_j$, w których są jakieś serwery zachodzi
\[
	|D(v_i) - D(v_j)| \leq d(v_i, v_j)
\]
\end{lemma}

\begin{proof}
Udowodnimy ten lemat przez indukcję względem liczby kroków. 
Na początku $D(\cdot) \equiv 0$ i nierówność jest trywialnie spełniona.
Załóżmy, że nierówność jest spełniona do początku pewnego kroku i pokażemy, że zachodzi też na jego końcu.
Bez straty ogólności możemy założyć, że na początku kroku serwera nie ma w punkcie $v_0$. Jeśli żądanie przychodzi
z wierzchołka innego niż $v_0$, $\BAL$ nie przesuwa żadnego ze swoich serwerów; możemy zatem założyć, że 
żądanie jest w wierzchołku $v_0$. Załóżmy, że $\BAL$ przesuwa do $v_0$ serwer, który był do tej pory w $v_\ell$. 

Dla porządku będziemy oznaczać wartości $D$ na końcu kroku przez $D'$.
Zauważmy, że dla $i \neq 0,\ell$ wartość $D'(v_i)$ jest taka sama jak $D(v_i)$ oraz $D'(v_0) = D(v_\ell) + 
d(v_0, v_\ell)$.\footnote{$D(v_0)$ było niezdefiniowane, na końcu kroku $D'(v_\ell)$ jest 
niezdefiniowane.} Zatem wystarczy pokazać, że na końcu kroku $|D'(v_0) - D'(v_j)| \leq d(v_0, v_j)$ zachodzi
dla dowolnego $j \neq 0,\ell$.

Z założenia indukcyjnego oraz nierówności trójkąta otrzymujemy
\begin{align}
\label{eq:distance_bounded_1}
	D'(v_j) - D'(v_0) &\;=\; D(v_j) - D(v_\ell) - d(v_0, v_\ell) \;\leq\; d(v_j,v_\ell) - d(v_0, v_\ell)
	\;\leq\; d(v_j, v_0) 
	\enspace.
%\end{align*}
\intertext{Z drugiej strony}
%\begin{align*}
\nonumber
	D'(v_0) - D'(v_j) & \;=\; D(v_\ell) + d(v_0, v_\ell) - D(v_j) \enspace.
\intertext{
Ponieważ $v_\ell$ został wybrany przez algorytm $\BAL$ do przesunięcia, musiała zachodzić nierówność 
$D(v_\ell) + d(v_\ell,v_0) \leq D(v_j) + d(v_j,v_0)$. 
Zatem}
\label{eq:distance_bounded_2}
	D'(v_0) - D'(v_j) & \;\leq\; D(v_j) + d(v_0, v_j) - D(v_j) = d(v_0,v_j) \enspace.
\end{align}
Z nierówności \ref{eq:distance_bounded_1} i \ref{eq:distance_bounded_2} wynika teza lematu.
\end{proof}

%%%%%%%%%%%%%%%%%%%%%%%%%%%%%%%%%%%%%

Zauważmy, że $\BAL$ zawsze pokrywa swoimi serwerami $k$ wierzchołków. Niepokryty wierzchołek nazywamy 
{\em dziurą}. Wierzchołek, w którym $\BAL$ ma obecnie dziurę będziemy oznaczać przez $\curr$ a wierzchołek,
w którym $\BAL$ miał poprzednio dziurę nazywamy $\prev$. Oznacza to, że ostatnim przesunięciem jakie 
$\BAL$ wykonał były przenosiny serwera z wierzchołka $\curr$ do wierzchołka $\prev$.
Przyglądając się bliżej nierówności \ref{eq:distance_bounded_1} można zauważyć następującą rzecz.

\vspace{0.3cm}
\begin{observation}
\label{obs:balance}
Dla dowolnego $v_j$, w którym jest serwer (tj.~różnym od $\curr$) zachodzi
\[
	D(v_j) - D(\prev) \leq d(v_j,\curr) - d(\curr,\prev) \enspace.
\]
\end{observation}

\begin{proof}
Jeśli $v_j = \prev$, to powyższa nierówność jest spełniona trywialnie. W przeciwnym przypadku 
obserwacja wynika z nierówności \ref{eq:distance_bounded_1} po podstawieniu $\prev$ pod $v_\ell$ i 
$\curr$ pod~$v_0$.
\end{proof}

%%%%%%%%%%%%%%%%%%%%%%%%%%%%%%%%%%%%%


\vspace{0.3cm}
\begin{theorem}
Algorytm $\BAL$ jest $k$-konkurencyjny dla dowolnych $k+1$-punktowych przestrzeni metrycznych.  
\end{theorem}

\begin{proof}
Zauważmy, że istnieje algorytm optymalny, który jest leniwy, zatem 
$\OPT$ ma też jedną dziurę. Niech $S = \sum_{v_i \neq \curr} D(v_i)$. 
Zdefiniujmy następującą funkcję potencjału 
\[
	\Phi = \begin{cases}
		k \cdot (D(\prev) - d(\curr,\prev)) - S & \textnormal{jeśli dziura $\OPT$ jest w $\curr$} \enspace, \\
		k \cdot D(v) - S & \textnormal{jeśli dziura $\OPT$ jest w $v \neq \curr$} \enspace.
	\end{cases}
\]
Zauważmy, że funkcja $\Phi$ nie jest zdefiniowana dopóki $\BAL$ nie wykona pierwszego przesunięcia serwera
(załóżmy, że zachodzi to w kroku $t_0$).
Dodatkowo w przeciwieństwie do poprzednio rozważanych funkcji potencjału $\Phi$ niekoniecznie jest nieujemna. 
Jednak na mocy lematu \ref{lem:balance}, $\Phi$ jest ograniczona od dołu przez wielkości niezależne od 
sekwencji wejściowej. Zatem istnieje takie $F_1$ (będące funkcją odległości punktów w metryce), 
że $\Phi \geq F_1$. 

Koszt algorytmu $\BAL$ do momentu pierwszego przesunięcia serwera jest ograniczony; 
oznaczmy to ograniczenie przez $F_2$.
Jeśli pokażemy, że w każdym kroku $t \geq t_0$ zachodzi 
\begin{equation}
\label{eq:bal_competitive}
\BAL(t) + \Delta\Phi(t) \leq k \cdot \OPT(t) \enspace,
\end{equation}
to otrzymamy, że dla dowolnej sekwencji wejściowej $\sigma$ zachodzi
\[ \BAL(\sigma) \leq k \cdot \OPT(\sigma) + F_2 - F_1 \enspace,\]
czyli że algorytm $\BAL$ jest $k$-konkurencyjny (choć niekoniecznie ściśle $k$-konkurencyjny).

Zauważmy, że jeśli odwołanie jest w wierzchołku różnym od dziury $\curr$, to $\BAL$ nie przesuwa żadnego 
z serwerów, jego koszt jest równy zero, wierzchołki $\curr$, $\prev$ i wartości $D(\cdot)$ pozostają takie 
same, a więc $\Delta\Phi = 0$. Koszt algorytmu optymalnego jest oczywiście nieujemny, a zatem
nierównośc $\ref{eq:bal_competitive}$ jest trywialnie spełniona.

Poniżej będziemy zatem zakładać, że żądanie występuje w wierzchołku $\curr$. 
Rozbijemy (tak, znowu) operacje w kroku $t \geq 2$ na dwie akcje.

\begin{description}
\item[\textnormal{\em Akcja 1, ruch $\OPT$.}] 
	Jeśli na początku ruchu $\OPT$ nie ma dziury w wierzchołku w wierzchołku $\curr$, to ponieważ jest 
	algorytmem leniwym, to nic nie robi, potencjał pozostaje niezmieniony 
	i nierówność \ref{eq:bal_competitive} jest trywialnie spełniona.
	Załóżmy więc, że $\OPT$ przesuwa jeden ze swoich serwerów  
	z wierzchołka $v_j$ do wierzchołka $\curr$, płacąc $d(v_j,\curr)$.
	Pokażemy, że potencjał wzrasta co najwyżej o $k \cdot d(v_j,\curr)$. 

	Przed ruchem potencjał jest równy $k \cdot (D(\prev) - d(\curr,\prev)) - S$,
	natomiast po ruchu jest równy $k \cdot D(v_j) - S$. Zatem
	\[
		\Delta\Phi = k \cdot \left(\,D(v_j) - D(\prev) + d(\curr,\prev)\,\right) 
		\leq k \cdot d(v_j,\curr) \enspace,
	\]
	gdzie ostatnia nierówność wynika z obserwacji \ref{obs:balance}.
	
\item[\textnormal{\em Akcja 2, ruch $\BAL$.}]
	Na początku tej akcji $\OPT$ ma już serwer w punkcie $\curr$ (a jego dziura --- używając 
	oznaczenia z poprzedniego przypadku --- jest w punkcie $v_j$) 
	Stąd potencjał na początku tej akcji wynosi $\Phi_\textnormal{B} = k \cdot D(v_j) - S$.
	Niech $S'$ oznacza wartość $S$ po ruchu $\BAL$; 
	łatwo zauważyć, że $S' = S + d(v_\ell, \curr)$, gdzie $v_\ell$ jest pozycją serwera, który jest
	przesuwany do wierzchołka $\curr$. Z tego przesunięcia wynika, że 
	\[
		D'(\curr) = D(v_\ell) + d(v_\ell,\curr) \enspace.
	\]
	Dodatkowo po przesunięciu wierzchołki $\prev$ i $\curr$ zmieniają się:
	\begin{align*}
		\curr' = v_\ell & & \prev' = \curr 
	\end{align*}
	%
	Jeśli $v_\ell = v_j$, nowa dziura algorymu $\BAL$ tworzy się w tym samym miejscu co dziura $\OPT$.
	Wtedy potencjał na końcu ruchu wynosi 
	\begin{align*}
		\Phi_\textnormal{F} \;
		 & =\; k \cdot (D'(\prev') - d(\curr', \prev')) - S' \\
		 & =\; k \cdot (D'(\curr) - d(v_\ell, \curr)) - S'  \\
		 & =\; k \cdot D(v_\ell) - (S+d(v_\ell,A)) \\
		 & =\; \Phi_\textnormal{B} - d(v_\ell,A) 
		\enspace.
	\end{align*}
	%
 	W drugim przypadku $v_\ell \neq v_j$.  Wtedy potencjał na końcu wynosi $\Phi_\textnormal{F} = 
	k \cdot D(v_j) - S'$, czyli również jest mniejszy od $\Phi_\textnormal{B}$ o $d(v_\ell,A)$.

	Pokazaliśmy więc, że zmniejszenie się potencjału pokrywa koszt algorytmu.

\end{description}
Zatem w każdym przypadku nierówność \ref{eq:bal_competitive} zachodzi.
\end{proof}

