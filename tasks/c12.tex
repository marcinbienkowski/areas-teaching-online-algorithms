\documentclass[a4paper,11pt]{article}
\usepackage[margin=2cm,bottom=2cm]{geometry}
\usepackage{amsmath,amsthm,amsopn,amsfonts}
\usepackage{polski}
\usepackage[utf8]{inputenc}
\usepackage{enumerate}
\usepackage{graphicx}
\pagestyle{empty}

\newcommand{\makeheader}[1]{
\begin{center}
{\Large\bf\sectfont Algorytmy online\\}
\vspace{0.3cm}
{\large\bf\sectfont Lista #1}\\
\end{center}
\vspace{0.7cm}
}


\newcommand{\myfigure}[3]{
\begin{figure}[t]
\centering
\fboxsep7pt
\framebox[0.95\textwidth]{
    \begin{minipage}{0.95\textwidth}
        \centering #3
    \end{minipage}
}
\caption{#2}
\label{fig:#1}
\end{figure}
}



\newcommand{\makefooter}{
\bigskip
\hfill {\em Marcin Bieńkowski}
}

\newcommand{\E}{\mathbf{E}}
\newcommand{\ALG}{\textsc{Alg}}
\newcommand{\OPT}{\textsc{Opt}}
\newcommand{\e}{\textnormal{e}}
\newcommand{\CNT}{\textsc{Cnt}}


\begin{document}
\makeheader{12}


\begin{description}

\item[Zadanie 1.] \textbf{(2 pkt.)} 
W problemie drzewa Steinera dany jest $n$-wierzchołkowy graf z wyróżnionym
wierzchołkiem $r$. Wejście jest ciągiem wierzchołków grafu; jeśli na wejściu
pojawia się wierzchołek $v$, algorytm musi zwrócić pewną ścieżkę łączącą $r$ z $v$.
W każdym momencie rozwiązaniem algorytmu jest zatem pewne drzewo łączące $r$ i 
wszystkie żądania, zaś jego kosztem jest sumaryczna waga krawędzi tego drzewa.
Skonstruuj randomizowany $O(\log n)$-konkurencyjny algorytm dla tego problemu.

\emph{Wskazówka:} Skonstruuj $1$-konkurencyjny algorytm online \ALG, gdy grafu będącego drzewem. 
Przybliż graf losowym drzewem konstrukcją FRT z wykładu i uruchom na tym drzewie \ALG.

\item[Zadanie 2.] \textbf{(2 pkt.)} 
Mamy do dyspozycji $k$-konkurencyjny algorytm dla problemu $k$ serwisantów dla 
drzew.\footnote{Przedstawiony na wykładzie algorytm \textsc{Double Coverage} 
można łatwo rozszerzyć do drzew: do żądania przesuwa on ze stałą prędkością 
``niezasłoniętych'' serwisantów. (Serwisant $x$ jest niezasłonięty jeśli ścieżka 
od żądania do $x$ nie zawiera innych serwisantów).}
Dany jest $n$-wierzchołkowy graf. Skonstruuj randomizowany 
$O(k \cdot \log n)$-konkurencyjny algorytm dla tego grafu.

\item[Zadanie 3.] \textbf{(2 pkt.)}
Drzewo $T$ otrzymane w przedstawionej na wykładzie konstrukcji FRT 
składa się z liści odpowiadających
wierzchołkom grafu $V$ i wierzchołków wewnętrznych $V'$. Pokaż, jak na
podstawie $T$ skonstruować drzewo $T'$, w którym zbiór wierzchołków jest równy $V$ a
odległości w~$T'$ między dowolną parą $v,w \in V$ różnią się od odległości
w $T$ co najwyżej $O(1)$ razy. \emph{Wskazówka:} usuwaj niektóre krawędzie.

\item[Zadanie 4.]
W pewnym kraju linia brzegowa jest prostą oddzielającą brzeg od morza. Pewien
kapitan odpłynął od brzegu na odległość 1 km, rzucił kotwicę i zrobił wielką
balangę. Kiedy następnego dnia obudził się na kacu, nie wiedział skąd ani jak 
się tu wziął; pamiętał tylko, że od odpłynął od brzegu na odległość 1 km. 
Oczywiście panuje gęsta mgła a kapitan jest głuchy i ślepy. Na pokładzie nie ma krowy,
która mogłaby pomóc znaleźć mu brzeg, jest to zatem zadanie dla Ciebie.
(Jest to wariant poszukiwania prostej na płaszczyźnie: tym razem wiadomo tylko, że
prosta jest w odległości równej~$1$). 

\begin{enumerate}
\item \textbf{(2 pkt.)} Skonstruuj deterministyczny algorytm, którego współczynnik 
	ścisłej konkurencyjności wynosi dokładnie $1 + 2\cdot \pi$. 
\item \textbf{(2 pkt.)} Skonstruuj deterministyczny algorytm, którego współczynnik 
	ścisłej konkurencyjności jest mniejszy niż $1 + 2\cdot \pi$. 
\end{enumerate}



\end{description}

\makefooter
\end{document}

