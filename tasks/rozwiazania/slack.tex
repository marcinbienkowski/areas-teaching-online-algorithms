\documentclass[a4paper,10pt]{article}
\usepackage[margin=2cm,bottom=3cm]{geometry}
\usepackage{amsmath,amsopn,amsfonts,amsthm}
\usepackage{lmodern}
\usepackage[T1]{fontenc}
\usepackage[polish]{babel}
\usepackage[utf8]{inputenc}
\usepackage[noend]{algpseudocode}
\usepackage{enumerate}
\usepackage{ifpdf}
\usepackage{graphicx}
\usepackage{paralist}
\usepackage{marginnote}

%%%%%%%%%%%%%%%%%%%%%%%%%%%%%%%%%%%%%%%
% Theorem-style environments
%%%%%%%%%%%%%%%%%%%%%%%%%%%%%%%%%%%%%%%

\newtheorem{theorem}{Twierdzenie}[section]
\newtheorem{lemma}[theorem]{Lemat}
\newtheorem{corollary}[theorem]{Wniosek}
\newtheorem{definition}[theorem]{Definicja}
\newtheorem{note}[theorem]{Uwaga}
\newtheorem{observation}[theorem]{Obserwacja}
\newtheorem{exercise}[theorem]{Ćwiczenie}
\newtheorem{example}[theorem]{Przykład}

%\newtheoremstyle{note}{6pt}{6pt}{\itshape}{}{\itshape}{.}{.5em}{}
%\theoremstyle{note}
%\newtheorem{observation}[theorem]{Obserwacja}
%\theoremstyle{plain}

\def\proofname{Dowód}
\renewcommand{\openbox}{\leavevmode\rule{1.4ex}{1.4ex}}
\makeatletter
 \renewenvironment{proof}[1][\proofname]{\par
 \pushQED{\qed}%
 \normalfont \topsep6\p@\@plus6\p@\relax
 \trivlist
 \item[\hskip\labelsep
 \bfseries
 #1\@addpunct{.}]\ignorespaces
 }{%
 \popQED\endtrivlist\@endpefalse
 }
\makeatother

\newcommand{\problem}[2]{
\vspace{0.15cm}
\noindent
\fboxsep7pt
\framebox[\textwidth]{
\begin{minipage}{0.95\textwidth}
\textsc{#1}. #2
\end{minipage}
}
\medskip
}


\newcommand{\myfigure}[3]{
\begin{figure}[t]
\centering
\fboxsep7pt
\framebox[0.95\textwidth]{
	\begin{minipage}{0.95\textwidth}
		\centering #3
	\end{minipage}
}
\caption{#2}
\label{fig:#1}
\end{figure}
}

\newcommand{\myfigurehere}[2]{
\begin{figure}[h]
\centering
\fboxsep7pt
\framebox[0.95\textwidth]{
	\begin{minipage}{0.95\textwidth}
		\centering #2
	\end{minipage}
}
\label{fig:#1}
\end{figure}
}


\newcommand{\rozdzial}[1]{
\input n#1
\vspace{0.3cm}
\newpage
}

\newcommand{\etalchar}[1]{$^{#1}$}

\newcommand{\myparagraph}[1]{%
\medskip
\noindent
{\bf #1.}
}


%%%%%%%%%%%%%%%%%%%%%%%%%%%%%%%%
% Commands 
%%%%%%%%%%%%%%%%%%%%%%%%%%%%%%%%

% editorial 

\newcommand{\todo}[1]{\noindent\colorbox{red}{TODO: #1}}
\newcommand{\COMMENTED}[1]{}

% algorytmy generic

\newcommand{\ADV}{\textsc{Adv}}
\newcommand{\ALG}{\textsc{Alg}}
\newcommand{\RAND}{\textsc{Rand}}
\newcommand{\OPT}{\textsc{Opt}}
\newcommand{\DET}{\textsc{Det}}
\newcommand{\COST}{\textsc{Cost}}

% algorytmy rozne

\newcommand{\RMTF}{\textsc{Rand-MTF}}
\newcommand{\HMTF}{\textsc{Half-MTF}}
\newcommand{\FWF}{\textsc{Fwf}}
\newcommand{\LRU}{\textsc{Lru}}
\newcommand{\FC}{\textsc{FC}}
\newcommand{\MARK}{\textsc{Mark}}
\newcommand{\RMARK}{\textsc{R-Mark}}
\newcommand{\LFD}{\textsc{Lfd}}
\newcommand{\LIFO}{\textsc{Lifo}}
\newcommand{\FIFO}{\textsc{Fifo}}
\newcommand{\LFU}{\textsc{Lfu}}
\newcommand{\TRANS}{\textsc{Trans}}
\newcommand{\MTF}{\textsc{Mtf}}
\newcommand{\BIT}{\textsc{Bit}}
\newcommand{\FF}{\textsc{First-Fit}}
\newcommand{\FFS}{\textsc{FF}}
\newcommand{\GREEDY}{\textsc{Greedy}}
\newcommand{\WFA}{\textsc{Wfa}}
\newcommand{\CRS}{\textsc{CRS}}
\newcommand{\EXP}{\textsc{Exp}}
\newcommand{\FLIP}{\textsc{Flip}}
\newcommand{\MTM}{\textsc{Mtm}}
\newcommand{\MTLM}{\textsc{Mtlm}}
\newcommand{\STAT}{\textsc{Stat}}
\newcommand{\CNT}{\textsc{Count}}
\newcommand{\EDGE}{\textsc{Edge}}
\newcommand{\DC}{\textsc{DC}}
\newcommand{\SC}{\textsc{SC}}
\newcommand{\CIRC}{\textsc{Circ}}
\newcommand{\CS}{\textsc{CostShare}}
\newcommand{\SCS}{\textsc{SamplingCostShare}}


\newcommand{\vflip}{v_\mathrm{FLIP}}
\newcommand{\vopt}{v_\mathrm{OPT}}
\newcommand{\PMTM}{P_\mathrm{MTM}}
\newcommand{\PMTLM}{P_\mathrm{MTLM}}
\newcommand{\POPT}{P_\mathrm{OPT}}
\newcommand{\PCNT}{P_\mathrm{COUNT}}
\newcommand{\w}{\mathcal{w}}
\newcommand{\Mmin}{M_\mathrm{min}}

% rozne

\newcommand{\I}{\mathcal{I}}
\newcommand{\A}{\mathcal{A}}
\renewcommand{\O}{\mathcal{O}}
\newcommand{\E}{\mathbf{E}}
\newcommand{\R}{\mathcal{R}}
\newcommand{\U}{\mathcal{U}}
\newcommand{\NAT}{\mathbb{N}}
\newcommand{\REAL}{\mathbb{R}}
\renewcommand{\Pr}{{\bf Pr}}
\newcommand{\eps}{\varepsilon}

\renewcommand{\P}{\mathcal{P}}
\newcommand{\T}{\mathcal{T}}
\newcommand{\F}{\mathcal{F}}
\newcommand{\bb}{\mathbf{b}}

\newcommand{\e}{\mathrm{e}}


\begin{document}

\subsection{Algorytm Slack Coverage}

W tej sekcji zdefiniujemy algorytm \textsc{SlackCoverage (SC)}  działający dla $2$
serwerów na płaszczyźnie i pokażemy, że jest on $3$-konkurencyjny. Dla
dowolnego $\gamma \in [0,1]$ mamy:

\begin{algorithmic}
\State Serwery algorytmu są w $x$, $y$; serwer $x$ jest tym, który jest bliższy $\sigma_t$. 
\State Niech $A$, $X$, $Y$ będą zdefiniowane jak na rysunku \ref{fig:slack_coverage}a. 
\State $s \gets \gamma \cdot (X + A - Y)$. 
\State Przesuń $y$ o $s$ w stronę $x$. 
\State Przesuń $x$ do $\sigma_t$.
\end{algorithmic}

\myfigure{slack_coverage}{Pierwsza i druga częśc kroku algorytmu {\sc SlackCoverage}}{\includegraphics{pict/slack-coverage}}

Zanim udowodnimy konkurencyjność algorytmu $\SC$ zauważmy, że $\SC_{1/2}$
zastosowany na linii jest równoważny algorytmowi $\DC$. 

\vspace{0.3cm}
\begin{theorem}[\cite{slack-coverage}]
Algorytm $\SC_{1/2}$ jest $3$-konkurencyjny.
\end{theorem}

\begin{proof}
W dowodzie wykorzystamy podobną funkcję jak w dowodzie konkurencyjności algorytmu $\DC$.
Niech $a$ i $b$ będą dowolnymi nieujemnymi liczbami rzeczywistymi, których wartości ustalimy później.
Niech 
\[
	\Phi = a \cdot \Mmin + b \cdot d(x,y) \enspace,
\]
gdzie $\Mmin$ jest ponownie kosztem minimalnego skojarzenia między serwerami
$\SC$ a serwerami $\OPT$.  Jak w poprzednim dowodzie dzielimy krok $t$ na dwie
części i dla każdej z nich chcemy pokazać, że zamortyzowany koszt algorytmu
jest nie większy niż koszt $\OPT$ razy współczynnik konkurencyjności.

W pierwszej części $\OPT$ przesuwa swoje serwery. 
Przy przesunięciu dowolnego z nich na odległość $d$, koszt minimalnego skojarzenia może 
wzrosnąć o $d$, a zatem potencjał może wzrosnąć o $a \cdot \OPT$. 
Zatem w pierwszej części kroku algorytm jest $a$-konkurencyjny. 

Dla pokazania konkurencyjności w drugiej części kroku obliczmy najpierw koszt algorytmu $\SC$. 
$\SC$ płaci $X$ za przesunięcie serwera $x$ oraz $s$ za przesunięcie serwera $y$. 
\[
	\SC = X + s \enspace.
\]
Jaka jest zmiana potencjału związana ze zmianą odległości między $x$ i $y$? 
Na początku kroku ta odległość wynosiła $A$. Serwer $y$ przesuwa się o 
$\gamma \cdot (X+A-Y) \leq \gamma \cdot A$ po prostej łączącej $x$ i $y$, przybliżając się do 
$\sigma_t$. Niech $y'$ oznacza nową pozycję serwera $y$, tak jak na rysunku \ref{fig:slack_coverage}b. 
Zatem $d(y',\sigma_t) \leq d(y,\sigma_t) = Y$. Stąd otrzymujemy
\[
	\Delta d(x,y) \leq Y - A \enspace.
\]

Jaka jest zmiana wartości $\Mmin$? Na początku drugiej części kroku $\OPT$ musi mieć jeden z serwerów 
(nazwijmy go $opt_1$) w punkcie $\sigma_t$. Nazwijmy drugi z serwerów $opt_2$.
Rozważmy dwa przypadki. 
\begin{description}
\item[\textnormal{\em Przypadek 1}.] Na początku drugiej części kroku $x$ jest skojarzony z $\sigma_t$.
	Wtedy ruch $x$ powoduje zmniejszenie się kosztu skojarzenia o $X$, a ruch $y$ powoduje zwiększenie się 
	$\Mmin$ o co najwyżej $s$. Stąd \[ 
	\Delta\Mmin \leq s - X \enspace.
	\]
	Zatem dla dowodu konkurencyjności wystarczy, żeby
	\begin{equation}
	\label{eq:sc_case_1}
	X + s + a \cdot (s - X) + b \cdot (Y - A) \leq 0 \enspace
	\end{equation}

\item[\textnormal{\em Przypadek 2}.] Na początku drugiej części kroku $x$ jest skojarzony z $opt_2$
	(a więc $y$ jest skojarzony z $\sigma_t$). Zatem początkowa wartość $\Mmin$ to $d(x,opt_2) + Y$.
	Po przesunięciu serwerów w minimalnym skojarzeniu $x$ zostaje skojarzony z $\sigma_t$, 
	a $y$ z $opt_2$. Zatem końcowa wartość $\Mmin$ to $d(y', s_2) \leq (A-s) + d(x,opt_2)$.
	Stąd w tym przypadku 
	\[
		\Delta\Mmin \leq A-s - Y \enspace.
	\]	
	Zatem dla dowodu konkurencyjności wystarczy, żeby
	\begin{equation}
	\label{eq:sc_case_2}
	X + s + a \cdot (A - s - Y) + b \cdot (Y - A) \leq 0 \enspace 
	\end{equation}

\end{description}
Łatwo sprawdzić, że dla $\gamma = 1/2$, $a = 3$ i $b = 2$ nierówności \ref{eq:sc_case_1} i 
\ref{eq:sc_case_2} są spełnione, a zatem algorytm $\SC$ jest $a$-konkurencyjny.
\end{proof}

\end{document}

%%%%%%%%%%%%%%%%%%%%%%%%%%%%%%%%%%%%%

