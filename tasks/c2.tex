\documentclass[a4paper,11pt]{scrartcl}
\usepackage[margin=2cm,bottom=2cm]{geometry}
\usepackage{amsmath,amsthm,amsopn,amsfonts}
\usepackage{polski}
\usepackage[utf8]{inputenc}
\usepackage{enumerate}
\usepackage{graphicx}
\pagestyle{empty}

\newcommand{\makeheader}[1]{
\begin{center}
{\Large\bf\sectfont Algorytmy online\\}
\vspace{0.3cm}
{\large\bf\sectfont Lista #1}\\
\end{center}
\vspace{0.7cm}
}


\newcommand{\myfigure}[3]{
\begin{figure}[t]
\centering
\fboxsep7pt
\framebox[0.95\textwidth]{
    \begin{minipage}{0.95\textwidth}
        \centering #3
    \end{minipage}
}
\caption{#2}
\label{fig:#1}
\end{figure}
}



\newcommand{\makefooter}{
\bigskip
\hfill {\em Marcin Bieńkowski}
}

\newcommand{\E}{\mathbf{E}}
\newcommand{\ALG}{\textsc{Alg}}
\newcommand{\OPT}{\textsc{Opt}}
\newcommand{\e}{\textnormal{e}}
\newcommand{\CNT}{\textsc{Cnt}}


\begin{document}
\makeheader{2}

\newcommand{\LFU}{\textsc{Lfu}}
\newcommand{\RMARK}{\textsc{R-Mark}}


\begin{description}

\item[Zadanie 1] 
Załóżmy, że algorytm dysponuje pamięcią podręczną o rozmiarze $k$,
a optymalny algorytm pamięcią o rozmiarze $h \leq k$.
\begin{enumerate}[a)]
\item \textbf{(1 pkt.)} Pokaż, że dowolny deterministyczny algorytm zaznaczający jest $\frac{k}{k-h+1}$-kon\-ku\-ren\-cyj\-ny.
\item \textbf{(3 pkt.)} Pokaż, że konkurencyjność dowolnego algorytmu deterministycznego (niekoniecznie zaznaczającego) wynosi 
co najmniej $\frac{k}{k-h+1}$.
\end{enumerate}

\item[Zadanie 2] \textbf{(2 pkt.)} 
{\LFU} (\emph{Least Frequently Used}) 
to algorytm, który z każdą stroną wiąże licznik określający, ile odwołań wystąpiło do tej strony.
Zapisanie strony w pamięci podręcznej lub wyrzucenie jej stamtąd nie powoduje zerowania takiego licznika.   
{\LFU} wyrzuca z pamięci podręcznej stronę, która była najrzadziej używana (o~najmniejszej wartości licznika).
Udowodnij, że {\LFU} nie jest konkurencyjny. 

\item[Zadanie 3] Rozważmy randomizowany algorytm zaznaczający {\RMARK} dla problemu pamięci podręcznej.
\begin{enumerate}[a)]
%\item \textbf{(1 pkt.)} (Ćwiczenie z wykładu). Załóżmy, że wejście $\sigma$ składa się z $\ell$ faz; niech $n_i$ oznacza liczbę nowych stron w fazie $i$ (zakładamy, że 
%$f_1 = k$). Pokaż, że $\OPT(\sigma) \geq (1/2) \cdot \sum_{i=1}^\ell n_i$.
\item \textbf{(1 pkt.)} Pokaż, że jeśli pamięć RAM zawiera $n = k+1$ stron, to algorytm 
{\RMARK} jest $H_k$-konkurencyjny. 
\item \textbf{(3 pkt.)} Pokaż, że dla dowolnego $k \geq 2$ można dobrać takie $n$, że konkurencyjność {\RMARK} jest większa niż $H_k$.
\end{enumerate}

\end{description}

\makefooter
\end{document}

