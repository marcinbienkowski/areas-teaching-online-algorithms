\documentclass[a4paper,11pt]{article}
\usepackage[margin=2cm,bottom=2cm]{geometry}
\usepackage{amsmath,amsthm,amsopn,amsfonts}
\usepackage{polski}
\usepackage[utf8]{inputenc}
\usepackage{enumerate}
\usepackage{graphicx}
\pagestyle{empty}

\newcommand{\makeheader}[1]{
\begin{center}
{\Large\bf\sectfont Algorytmy online\\}
\vspace{0.3cm}
{\large\bf\sectfont Lista #1}\\
\end{center}
\vspace{0.7cm}
}


\newcommand{\myfigure}[3]{
\begin{figure}[t]
\centering
\fboxsep7pt
\framebox[0.95\textwidth]{
    \begin{minipage}{0.95\textwidth}
        \centering #3
    \end{minipage}
}
\caption{#2}
\label{fig:#1}
\end{figure}
}



\newcommand{\makefooter}{
\bigskip
\hfill {\em Marcin Bieńkowski}
}

\newcommand{\E}{\mathbf{E}}
\newcommand{\ALG}{\textsc{Alg}}
\newcommand{\OPT}{\textsc{Opt}}
\newcommand{\e}{\textnormal{e}}
\newcommand{\CNT}{\textsc{Cnt}}


\begin{document}
\makeheader{1}


\begin{description}

\item[Zadanie 1.] \textbf{(1 pkt.)} 
%
Pokaż, że współczynnik ścisłej konkurencyjności dowolnego deterministycznego algorytmu dla problemu poszukiwania krowy na prostej wynosi co najmniej $5$. 

\item[Zadanie 2.] \textbf{(1 pkt.)} 
%
Przypomnijmy, że w problemie poszukiwania krowy na prostej zakładamy, że krowa stoi w odległości równej co najmniej $1$ od punktu startowego. Przedstawiony na wykładzie algorytm odchodził od punktu startowego, raz w lewo, raz w prawo, zaczynając od kierunku ,,w~lewo'', na kolejne odległości: $q$, $q \cdot r$, $q \cdot r^2$, $q \cdot r^3$, \ldots.  Dla przedstawionego na wykładzie algorytmu wybraliśmy $q = 1$ oraz $r = 2$.

Rozważ problem poszukiwania krowy na prostej i załóż, że w powyższym algorytmie wybieramy początkowy kierunek rzucając (sprawiedliwą) monetą.  Pokaż, że dla $q = 1$ i $r = 2$ taki algorytm jest ściśle $7$-konkurencyjny.

\item[Zadanie 3.] \textbf{(4 pkt.)} 
%
Załóż, że w powyższym algorytmie wybieramy początkowy kierunek rzucając (sprawiedliwą) monetą, a dodatkowo wybieramy $\theta$ z jednostajnym rozkładem z~przedziału $[0,1)$. Pokaż, że dla $q = r^\theta$ taki algorytm jest ściśle $(1 + (r+1)/\ln r)$-konkurencyjny. Numerycznie wyznacz $r$ minimalizujące ten współczynnik. 

\item[Zadanie 4.] \textbf{(4 pkt.)}
%
Niech 
\[
    R = \frac{1}{1-(1-1/B)^B}.
\]
Skonstruuj randomizowany algorytm dla problemu wypożyczania nart, który jest ściśle $R$-kon\-ku\-ren\-cyj\-ny.\footnote{To jest najlepszy możliwy do osiągnięcia współczynnik konkurencyjności, ale nie trzeba tego udowadniać.} Do czego dąży to wyrażenie przy~$B \to \infty$?

\emph{Wskazówki: Algorytm randomizowany jest w pełni opisywalny przez ciąg prawdopodobieństw $p_1,p_2,p_3,\ldots,p_\infty$, gdzie $p_i$ jest prawdopodobieństwem zakupu nart dnia $i$ oraz \mbox{$\sum_{i=1}^\infty p_i = 1$}. Ogranicz się do klasy algorytmów, dla których $p_i = 0$ dla $i > B$.  Jakie są możliwe strategie adwersarza przeciwko algorytmowi z takiej klasy? Pod warunkiem zastosowania danej strategii adwersarza: jaki będzie oczekiwany koszt algorytmu i jaki będzie koszt algorytmu optymalnego? Zauważ, że dla każdej takiej strategii musi zachodzić $\E[\ALG] \leq R \cdot \OPT$. Ogranicz się do algorytmów, dla których wszystkie te nierówności są równością i wyznacz stąd wartości $p_i$.}

\end{description}

\makefooter
\end{document}

