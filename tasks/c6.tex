\documentclass[a4paper,11pt]{article}
\usepackage[margin=2cm,bottom=2cm]{geometry}
\usepackage{amsmath,amsthm,amsopn,amsfonts}
\usepackage{polski}
\usepackage[utf8]{inputenc}
\usepackage{enumerate}
\usepackage{graphicx}
\pagestyle{empty}

\newcommand{\makeheader}[1]{
\begin{center}
{\Large\bf\sectfont Algorytmy online\\}
\vspace{0.3cm}
{\large\bf\sectfont Lista #1}\\
\end{center}
\vspace{0.7cm}
}


\newcommand{\myfigure}[3]{
\begin{figure}[t]
\centering
\fboxsep7pt
\framebox[0.95\textwidth]{
    \begin{minipage}{0.95\textwidth}
        \centering #3
    \end{minipage}
}
\caption{#2}
\label{fig:#1}
\end{figure}
}



\newcommand{\makefooter}{
\bigskip
\hfill {\em Marcin Bieńkowski}
}

\newcommand{\E}{\mathbf{E}}
\newcommand{\ALG}{\textsc{Alg}}
\newcommand{\OPT}{\textsc{Opt}}
\newcommand{\e}{\textnormal{e}}
\newcommand{\CNT}{\textsc{Cnt}}


\begin{document}
\makeheader{6}


\begin{description}

\item[Zadanie 1.] Rozważmy problem szeregowania na $m$ identycznych maszynach (o takiej samej prędkości).
Celem jest minimalizacja obciążenia najbardziej obciążonej maszyny.

\begin{enumerate}

\item \textbf{(2 pkt.)} 
Pokaż, że strategia zachłanna (przypisujemy zadanie do najmniej obciążonej maszyny) ma współczynnik
ścisłej konkurencyjności równy $2-1/m$ (tj.~pokaż dolne i górne ograniczenie).

\item \textbf{(2 pkt.)} 
Pokaż, że dla dowolnego $m \geq 2$, ścisła konkurencyjność dowolnego randomizowanego algorytmu wynosi co najmniej $4/3$.
% optymalny lower bound = 4/3, działa dla m >= 2.
% dajemy m zadan wielkosci 1. Jesli ppb. ze algorytm ułoży je na m maszynach jest większe niż 2/3
% to dajemy dodatkowe zadanie wielkości 2.
% można też z yao: zadania 1,1 z ppb 1/2 i zadania 1,1,2 z ppb 1/2.

\end{enumerate}

\item[Zadanie 2.] \textbf{(2 pkt.)} 
Pokaż, że ścisła konkurencyjność problemu \textsc{Online Bidding} wynosi co najmniej~$4$. 
Możesz skorzystać z następującego schematu:
\begin{enumerate}
\item Ustal nieskończony ciąg $0 < d_1 < d_2 < d_3 \ldots $ definiujący algorytm i
oznacz konkurencyjność algorytmu przez $R$.
Zdefiniuj $s_j = \sum_{i=1}^j d_i$ i pokaż, że dla każdego $j$ 
musi zachodzić $s_{j+2} \leq R \cdot (s_{j+1} - s_j)$.

\item Zdefiniuj $b_j = s_{j+1} / s_j$ i pokaż, że 
\begin{equation}
	\label{eq:b_j}
	(b_{j+1} - b_j) \cdot b_j \leq -b_j^2 + R \cdot b_j - R
	\enspace.
\end{equation}

\item Pokaż, że dla $R < 4$ prawa strona \eqref{eq:b_j} jest ujemna, a zatem ciąg $\{b_j\}_j$ jest malejący. 

\item Zaobserwuj, że ciąg $\{b_j\}_j$ jest zbieżny (dlaczego?). Niech $b = \lim_{j \to \infty} b_j$. 
Do czego zbiega lewa i prawa strona \eqref{eq:b_j}?
\end{enumerate}

\item[Zadanie 3] \textbf{(2 pkt.)} 
%
Rozważamy szeregowanie zadań na $m$ powiązanych maszynach, a~funkcją kosztu jest maksymalne obciążenie maszyny. 
Przedstawiony na wykładzie algorytm $\textsc{SlowFit}(\lambda)$ ma następujące własności.
\begin{itemize}
\item Przypisuje zadania do maszyn, tak że maksymalne obciążenie wynosi co najwyżej $2 \lambda$. Być może dla pewnego zadania zamiast numeru maszyny algorytm zwróci komunikat~,,\texttt{FAIL}''.
\item Jeśli zostanie uruchomiony na wejściu $\sigma$, dla którego $\textsc{Opt}(\sigma) \leq \lambda$, to wszystkie zadania zostaną przypisane do maszyn (komunikat ,,\texttt{FAIL}'' nie wystąpi).
\end{itemize}

Pokaż, jak za jego pomocą skonstruować $8$-konkurencyjny algorytm.



\item[Zadanie 4.] \textbf{(2 pkt.)} 
%
(Problem inkrementalnych median). Dane są dwa zbiory: $K$ (klienci) i~$M$~(miejsca na zbudowanie fabryk) oraz
funkcja odległości $d(k,m)$ zdefiniowana dla dowolnej pary $(k,m) \in K \times M$.\footnote{Możesz założyć, że funkcja~$d$ spełnia nierówność trójkąta, choć nie jest to potrzebne w tym zadaniu.}
Mówimy, że zbiór fabryk $F \subseteq M$ ma koszt 
\[
	\textsc{cost}(F) \triangleq \sum_{k \in K} d(k,F) \triangleq \sum_{k \in K} \min_{f \in F} d(k,f)
	\enspace.
\]
Dla rozmiaru $s \in \{1,\ldots,|M|\}$ minimalny koszt zbioru $s$ fabryk
oznaczamy przez $\OPT(s)$. Przykładowo $\OPT(|M|)$ jest kosztem postawienia
fabryki w każdym dostępnym miejscu.

\newpage

Zadaniem algorytmu jest rozbudowa zbioru fabryk, 
tj.~skonstruowanie ciągu zbiorów $F_1 \subseteq F_2 \subseteq \ldots \subseteq F_{|M|-1} \subseteq F_{|M|}$,
takiego że dla każdego $i \in \{1,\ldots,|M|\}$ zachodzą warunki
\begin{enumerate}
\item $\textsc{cost}(F_i) \leq \OPT(i)$,
\item $|F_i| \leq 4 \cdot i$.
\end{enumerate}

\emph{Wskazówka: to zadanie jest proste, tylko ma długą treść. Być może do konstrukcji ciągu $F_i$ konieczne
będzie rozwiązanie jakiegoś NP-trudnego problemu, ale nie należy się tym martwić.}

\end{description}

\makefooter
\end{document}

