\documentclass[a4paper,11pt]{article}
\usepackage[margin=2cm,bottom=2cm]{geometry}
\usepackage{amsmath,amsthm,amsopn,amsfonts}
\usepackage{polski}
\usepackage[utf8]{inputenc}
\usepackage{enumerate}
\usepackage{graphicx}
\pagestyle{empty}

\newcommand{\makeheader}[1]{
\begin{center}
{\Large\bf\sectfont Algorytmy online\\}
\vspace{0.3cm}
{\large\bf\sectfont Lista #1}\\
\end{center}
\vspace{0.7cm}
}


\newcommand{\myfigure}[3]{
\begin{figure}[t]
\centering
\fboxsep7pt
\framebox[0.95\textwidth]{
    \begin{minipage}{0.95\textwidth}
        \centering #3
    \end{minipage}
}
\caption{#2}
\label{fig:#1}
\end{figure}
}



\newcommand{\makefooter}{
\bigskip
\hfill {\em Marcin Bieńkowski}
}

\newcommand{\E}{\mathbf{E}}
\newcommand{\ALG}{\textsc{Alg}}
\newcommand{\OPT}{\textsc{Opt}}
\newcommand{\e}{\textnormal{e}}
\newcommand{\CNT}{\textsc{Cnt}}


\begin{document}
\makeheader{8}


\begin{description}

\item[Zadanie 1.] \textbf{(3 pkt.)} 
Rozważmy problem routingu na linii o $n$ wierzchołkach, gdzie każda krawędź ma
pojemność $U$. Pokaż dolne ograniczenie na ścisłą konkurencyjność dowolnego
algorytmu deterministycznego w wysokości $\Omega(n^{1/U})$.

% Trzeba wziąć takie k, że k^U < n <= (k+1)^U. Wtedy k^U <= n-1 a n-1 to liczba krawedzi na linii. 
% Da się wymusić współczynnik k, a k+1 >= n^(1/U)

%\item[Zadanie 2] 
%Chcemy za pomocą jednej transakcji zamienić jedngo bitcoina na złotówki w ciągu najbliższych 42 dni. 
%Każdego dnia dowiadujemy się, jaki jest kurs danego dnia; jest on 
%dowolną liczbą rzeczywistą z zakresu $[1,M]$, gdzie $M$ jest znane algorytmowi od początku.
%\begin{enumerate}
%\item \textbf{(1 pkt.)} Skonstruuj deterministyczny algorytm, który będzie  
%    ściśle $\sqrt{M}$-konkurencyjny.
%\item \textbf{(2 pkt.)} Skonstruuj randomizowany algorytm, który będzie  
%    ściśle $O(\log M)$-konkurencyjny.
%\end{enumerate}


\item[Zadanie 2.]

W problemie szeregowania zadań na $m$ \emph{niepowiązanych} maszynach
dany jest ciąg zadań, gdzie zadanie $t$ jest zdefiniowane wektorem 
$p_t = (p_{t,1}, p_{t,2}, \dots, p_{t,m})$: jeśli zadanie przypiszemy
maszynie $j$, to jej obciążenie $\ell_j$ zwiększy się o $p_{t,j}$. 
Celem jest minimalizacja $\max_{j=1}^m \ell_j$.
Tak jak w przypadku algorytmu dla powiązanych maszyn chcemy stworzyć procedurę
$\ALG_\lambda$, która będzie przypisywać zadania \emph{przy założeniu, że $\OPT \leq \lambda$}.


Niech $y_{t,j} = 1$ jeśli zadanie $t$ przypisujemy do maszyny $j$, zaś $0$ w przeciwnym przypadku.
W~kroku $k$ zdefiniujmy następujący program całkowitoliczbowy $\mathcal{P}^{\mathrm{int}}_k$, 
którego celem jest maksymalizacja liczby przypisanych zadań.
\begin{align*}
\textnormal{maksymalizuj:} & \sum_{t=1}^k \sum_{j=1}^m y_{t,j} \enspace,& \\
\textnormal{przy zachowaniu warunków: }
	& \sum_{j=1}^m y_{t,j} \leq 1 & \textnormal{dla każdego $t \in \{1,\ldots,k\}$,} \\
    & \sum_{t=1}^k p_{t,j} \cdot y_{t,j} \leq \lambda & \textnormal{dla każdego $j \in \{1,\ldots,m\}$,} \\
	& y_{t,j} \in \{0,1\} & \textnormal{dla każdego $t \in \{1,\ldots,k\}, j \in \{1,\ldots,m\}.$} 
\end{align*}

Niech $\mathcal{P}_k$ będzie liniową relaksacją $\mathcal{P}^{\mathrm{int}}_k$,
w którym warunek $y_{t,j} \in \{0,1\}$ 
został zastąpiony przez $y_{t,j} \geq 0$
(Warunek $y_{t,j} \leq 1$ jest zbędny, bo implikowany przez 
$\sum_{j=1}^m y_{t,j} \leq 1$).
Programem dualnym do $\mathcal{P}_k$ jest następujące zagadnienie minimalizacyjne $\mathcal{D}_k$:
\begin{align*}
\textnormal{minimalizuj: } & \sum_{j=1}^m \lambda \cdot x_j + \sum_{t=1}^k z_t \\
\textnormal{przy warunkach: }
	& z_t + p_{t,j} \cdot x_{j} \geq 1 & 
	\textnormal{dla każdego $t \in \{1,\ldots,k\}, j \in \{1,\ldots,m\}.$} \\
	& x_j \geq 0 & \textnormal{dla każdego $j \in \{1,\ldots,m\}$,} \\
    & z_t \geq 0 & \textnormal{dla każdego $t \in \{1,\ldots,k\}$.} 
\end{align*}

\paragraph{Algorytm $\textsc{ALG}_\lambda$.}
Na początku niech $x_j = 1/(2 \lambda \cdot m)$ dla każdego $j \in \{1, \dots, m\}$.
W kroku $k$ niech $\mathcal{S}_k = \{ j :
p_{k,j} \leq \lambda \}$. Następnie:
\begin{itemize}
\item Jeśli $\mathcal{S}_k = \emptyset$ lub jeśli $x_j > 1/\lambda$ dla pewnego $j$, to zwróć \texttt{FAIL}. 
\item W przeciwnym
przypadku niech $j^*$ będzie maszyną z $\mathcal{S}_k$ minimalizującą $p_{k,j^*} \cdot x_{j^*}$.
\begin{enumerate}
\item przypisz zadanie $k$ do maszyny $j^*$ 
\item $z_t \leftarrow 1 - p_{k,j^*} \cdot x_{j^*}$
\item $x_{j^*} \leftarrow x_{j^*} \cdot \left(1 + \frac{p_{k,j^*}}{2 \cdot \lambda} \right)$
\end{enumerate}
\end{itemize}

\begin{enumerate}
\item \textbf{(2 pkt.)} Załóżmy, że \ALG nie zwróci \texttt{FAIL}. Ustalmy $j$. Pokaż, że w dowolnym momencie zachodzi 
\[
		x_j \geq \frac{1}{2\lambda \cdot m} \cdot C^{\ell_j / \lambda}
\]
dla pewnego $C > 1$.\footnote{Wskazówka: hasło do wyszukiwarki ,,useful inequalities''.} 
Wykorzystaj tę nierówność do pokazania, że $\ell_j = O(\log m) \cdot \lambda$.

\item \textbf{(2 pkt.)} Załóżmy, że \ALG nie zwróci \texttt{FAIL}. Pokaż, że w dowolnego kroku $k$, \ALG generuje dopuszczalne rozwiązanie dla $\mathcal{D}_k$ i 
zachodzi
\[
	\ALG(\mathcal{D}_k)	= k + 1 - \lambda \cdot \sum_{j=1}^m x_j.
\]

\item \textbf{(3 pkt.)} Pokaż, że jeśli \ALG zwróci \texttt{FAIL}, to \OPT nie jest w stanie przypisać wszystkich zadań 
tak, żeby obciążenie każdej maszyny było co najwyżej $\lambda$.

%Pokaż, że na końcu dowolnego kroku $k$ zachodzi $\ALG(\mathcal{D}_k) \leq 
%O(\log d) \cdot \ALG(\mathcal{P}_k)$.
%\item \textbf{(3 pkt.)} Pokaż, że w dowolnym kroku $k$ generowane przez algorytm rozwiązania 
%$\ALG(\mathcal{P}_k)$ i $\ALG(\mathcal{D}_k)$ są dopuszczalne. Przydatna może być nierówność $\ln (1+y) \geq y/2$,
%która jest prawdziwa dla $y \in [0,2]$.
\end{enumerate}

\end{description}

\makefooter
\end{document}

