\section{Porównanie adwersarzy}



\subsection{Silny adwersarz adaptujący się}

Dotychczas rozważanego adwersarza, który nie widzi bitów losowych algorytmu
nazywamy adwersarzem nieświadomym (ang. {\em oblivious}).

Adwersarz adaptujący się tworzy sekwencję wejściową na podstawie
dotychczasowego zachowania algorytmu. Ponieważ algorytm może wykorzystywać bity
losowe, a wejście jest (deterministyczną) funkcją zachowania algorytmu, wejście
jest zmienną losową. 

Rozróżniamy zatem dwa typy adwersarzy.  {\em Silny adwersarz adaptujący się}
oblicza rozwiązanie dla wygenerowanej sekwencji na końcu; możemy zatem założyć,
że będzie to optymalne rozwiązanie dla wygenerowanej sekwencji. Mówimy zatem,
że zrandomizowany algorytm $\ALG$ jest $k$-konkurencyjny przeciwko silnemu
adwersarzowi adaptującemu się, jeśli istnieje taka stała~$\alpha$, że dla
każdego w ten sposób generowanego wejścia~$I$ zachodzi 
\[
	\E[\ALG(I)] \leq k \cdot \E[\OPT(I)] + \alpha \enspace,
\]
gdzie wartość oczekiwana jest liczona po wyborach losowych algorytmu.


\begin{theorem}
Rozważmy dowolny algorytm $\ALG$ dla SRP. Jego współczynnik ścisłej konkurencyjności przeciwko 
silnemu adwersarzowi adaptującemu się wynosi co najmniej $\R = 2 - 1/B$.
\end{theorem}

\begin{proof}
Dowód jest identyczny jak dla przypadku dolnego ograniczenia dla deterministycznego algorytmu.
Sekwencja $\sigma$ tworzona przez adwersarza zawsze kończy się w dniu, w którym algorytm decyduje się 
na zakup nart. Załóżmy, że następuje to dniu $T \in \NAT \cup \{\infty\}$, gdzie $T$ jest zmienną losową zależną
od wyborów losowych algorytmu.

Wtedy $\ALG(\sigma) = T-1+B$. Z drugiej strony $\OPT(\sigma) = \min\, \{ B, T \}$. 
Wystarczy udowodnić, że $\E[\ALG(\sigma)] \geq \R \cdot \E[\OPT(\sigma)]$. 
My potrafimy udowodnić silniejsze twierdzenie, a mianowicie 
\[ \ALG(\sigma) \geq \R \cdot \OPT(\sigma) \enspace. \]
Powyższa nierówność zmiennych losowych jest spełniona dla każdego wyboru zdarzenia elementarnego.
Wynika ona z prostej analizy dwóch przypadków ($B \geq T$ lub $B < T$).
\end{proof}


\begin{theorem}
Rozważmy problem w którym $\ALG$ ma zawsze skończony zbiór odpowiedzi. Jeśli $\ALG$
jest $R$-konkurencyjny przeciwko silnemu adwersarzowi adaptującemu się, to 
istnieje $R$-konkurencyjny deterministyczny algorytm. 
\end{theorem}


Adwersarz adaptujący się musi generować rozwiązanie dla danej
sekwencji w sposób {\em online}, wraz z algorytmem. Myślimy o tym, że adwersarz
dostarcza swój własny deterministyczny algorytm $\ADV$, który jest funkcją
dotychczas widzianego kawałka wejścia i przeszłych akcji algorytmu $\ALG$ (tj.
decyzję w kroku $n$ może podjąć na podstawie odpowiedzi algorytmu do kroku
$n-1$).  Mówimy, że algorytm $\ALG$ jest $k$-konkurenyjny przeciwko
adwersarzowi adaptującemu się, jeśli istnieje taka stała $\alpha$, że dla
każdego możliwego algorytmu adwersarza $\ADV$ i każdego generowanego w taki
sposób wejścia $I$ zachodzi 
\[
	\E[\ALG(I)] \leq k \cdot \E[\ADV(I)] + \alpha \enspace,
\]
gdzie wartość oczekiwana jest liczona po wyborach losowych algorytmu.

Jeśli weźmiemy dowolny algorytm $\ALG$ i przez 
$\R_\mathrm{N}(\ALG)$, $\R_\mathrm{A}(\ALG)$ i $\R_\mathrm{SA}(\ALG)$ 
oznaczymy osiągane przez niego współczynniki
konkurencyjności przeciwko adwersarzowi nieświadomemu, adaptującemu się i silnemu adaptującemu się, 
to zachodzi:
\[
	\R_\mathrm{N}(\ALG) \leq \R_\mathrm{A}(\ALG) \leq \R_\mathrm{SA}(\ALG)
	\enspace.
\]
Druga nierówność jest oczywista, zaś pierwsza wynika z tego, że adwersarz adaptujący się jest w stanie 
symulować dowolnego adwersarza nieświadomego 
generując sekwencję i rozwiązanie dla niej na samym początku a~następnie ignorując 
faktyczne zachowanie algorytmu.

Powyższa nierówność zachodzi dla każdego algorytmu randomizowanego. Weźmy zatem
dowolny problem online. Niech współczynniki konkurencyjności najlepszych algorytmów 
zrandomizowanych rozwiązujących ten problem to odpowiednio
$\R_\mathrm{N}$, $\R_\mathrm{A}$ i $\R_\mathrm{SA}$, 
a współczynnik konkurencyjności najlepszego algorytmu 
deterministycznego to $\R_\mathrm{DET}$. Wtedy otrzymujemy
\begin{equation}
\label{eq:adv_power}
	\R_\mathrm{N} \leq \R_\mathrm{A} \leq \R_\mathrm{SA} \leq \R_\mathrm{DET}
	\enspace.
\end{equation}


\begin{theorem}
Rozważmy dowolny algorytm $\ALG$ dla SRP. Jego współczynnik ścisłej konkurencyjności przeciwko 
adwersarzowi adaptującemu się wynosi co najmniej $\R = 2 - 1/B$.
\end{theorem}

\begin{proof}
Wprowadzamy takie same oznaczenia jak w poprzednim dowodzie. Jak poprzednio adwersarz będzie tworzyć wejście
$\sigma$ do dnia, w którym algorytm kupi narty.

Ten dowód jest trochę trudniejszy niż poprzedni, gdyż adwersarz musi zdecydować o swojej strategii w 
sposób {\em online}, nie znając wartości zmiennej losowej $T$. Jednak na podstawie kodu algorytmu
adwersarz może wyliczyć wartość $\E[T]$. Jeśli $\E[T] \geq B$, to strategia adwersarza brzmi
,,kup narty pierwszego dnia'',  a w~przeciwnym przypadku strategia to ,,zawsze pożyczaj narty''.

W pierwszym z przypadków otrzymujemy
\[
	\E[\ALG(\sigma)] \;=\; \E[T] - 1 + B \;\geq\; 2 B - 1 \;\geq\; \R \cdot B \;=\; \R \cdot \ADV(\sigma) \enspace.
\]
Natomiast w drugim
\[
	\E[\ALG(\sigma)] \;=\; \E[T] - 1 + B \;=\; \frac{\E[T] - 1 + B}{\E[T]} \cdot \E[T] 
		\;\geq\; \frac{B - 1 + B}{B} \cdot \E[T] \;\geq\; \R \cdot \E[\ADV(\sigma)] \enspace,
\]
co kończy dowód. 
\end{proof}


\marginnote{W.13A}

W tej części połączymy konkurencyjność algorytmów
przeciwko różnym typom adwersarzy w sposób bardziej precyzyjny niż ten
dany przez nierówności~(\ref{eq:adv_power}). 

Dla skrócenia zapisu, w tym rozdziale rozważymy definicję 
konkurencyjności zadaną przez funkcje liniowe. 
Niech $\gamma$ będzie funkcją liniową. Jeśli mówimy, 
że deterministyczny algorytm $\ALG$ jest $\gamma$-konkurencyjny, to tozumiemy przez to 
że dla dowolnego wejścia $I$, $\ALG(I) \leq \gamma(\OPT(I))$. 
W analogiczny sposób definiujemy konkurencyjność algorytmów zrandomizowanych; 
zauważmy, że funkcje liniowe są przemienne z wartością oczekiwaną.

\subsection{Związek między adwersarzami przeciwko algorytmom zrandomizowanym}

Na początku przyjrzymy się bliżej adwersarzom adaptującym się.  Adwersarz taki
składa się z dwóch części: deterministycznej 
części $Q$ dającej zapytania i deterministycznej części odpowiadającej
na te zapytania $S$, równolegle do algorytmu.  Wtedy algorytm jest
$\gamma$-konkurencyjny przeciwko adwersarzowi adaptującemu się, jeśli jest
$\gamma$-konkurencyjny przeciwko dowolnemu adwersarzowi $(Q,S)$ gdzie $S$ jest
algorytmem online, oraz jest $\gamma$-konkurencyjny przeciwko silnemu
adwersarzowi adaptującemu się, jeśli jest $\gamma$-konkurencyjny przeciwko
dowolnemu adwersarzowi $(Q,\OPT)$. 

\begin{theorem}
Niech $\ALG$ będzie zrandomizowanym algorytmem, który jest
$\gamma$-konkurencyjny przeciwko adwersarzowi adaptującemu się, a $\ALG'$
algorytmem, który jest $\delta$-kon\-ku\-ren\-cyj\-ny przeciwko adwersarzowi
nieświadomemu. Wtedy $\ALG$ jest $(\gamma \circ \delta)$-konkurencyjny przeciwko
silnemu adwersarzowi adaptującemu się.
\end{theorem}

\begin{proof}
Niech $(Q,\OPT)$ będzie dowolnym adwersarzem adaptive-offline. Pokażemy, że
$\ALG$ jest przeciwko niemu $(\gamma \circ \delta)$-konkurencyjny. 

Najpierw pokażemy że $\ALG$ jest $\gamma$-,,konkurencyjny'' przeciwko 
adwersarzowi $(Q,\ALG')$. Konkurencyjność jest tutaj w cudzysłowie, gdyż 
$\ALG'$ jest algorytmem zrandomizowanym i nie zdefiniowaliśmy co to znaczy.
$\ALG'$ jest pewnym rozkładem $\pi$ nad wszystkimi algorytmami deterministycznymi
$\{S_j\}_j$, z kolei $\ALG$ jest pewnym innym rozkładem prawdopodobieństwa nad wszystkimi
algorytmami deterministycznymi $\{ \DET_i \}_i$.

Ponieważ $\ALG$ jest $\gamma$-konkurencyjny przeciwko dowolnemu adwersarzowi
adaptującemu się, w szczególności jest $\gamma$-konkurencyjny przeciwko $(Q,S_j)$ 
dla dowolnego $j$, tj.:
\[
\E_i[\DET_i(\sigma(Q,\DET_i)] \leq \gamma (\E_i[S_j(\sigma(Q,\DET_i)]) \enspace.
\]
Mnożąc obie strony przez $\pi(j)$ i sumując po wszystkich $j$ dostajemy
\begin{align*}
\E_i[\DET_i(\sigma(Q,\DET_i)] 
	\;=\; & \sum\nolimits_j \pi(j) \cdot \E_i[\DET_i(\sigma(\DET_i,Q))] \\
	 \;\leq\; & \sum\nolimits_j \pi(j) \cdot \gamma (\E_i[S_j(\sigma(\DET_i,Q))]) \\
	 \;=\; & \E_j [ \gamma (\E_i[S_j(\sigma(\DET_i,Q))]) ] \\
	 \;=\; & \gamma( \E_i\E_j[S_j(\sigma(\DET_i,Q))])  \enspace,
\end{align*}
Z $\delta$-konkurencyjności algorytmu $\ALG'$ wynika, że 
$\E_j[S_j(\sigma)] \leq \delta (\OPT(\sigma))$ zachodzi dla dowolnego wejścia $\sigma$,
które nie zależy od wyborów losowych algorytmu $\ALG'$. 
Zatem 
\begin{align*}
\E_i[\DET_i(\sigma(\DET_i,Q))] 
	\;\leq\; & \gamma (\E_i[\delta(\OPT(\sigma(\DET_i,Q)))]) \\
	\;=\; & (\gamma \circ \delta)\; (\E_i[\OPT(\sigma(\DET_i,Q))]) \enspace. \qedhere
\end{align*}
\end{proof}


%%%%%%%%%%%%%%%%%%%%%%%%%%%%%%%%%%%%%%

\subsection{Silny adwersarz adaptujący się a algorytmy deterministyczne}

W tej części pokażemy, że randomizacja przeciwko silnemu adwersarzowi
adaptującemu się nie pomaga. Do poniższego dowodu potrzebne jest jednak jedno
techniczne założenie, że dla każdego żądania adwersarza zbiór możliwych
odpowiedzi jest skończony.

\begin{theorem}
\label{thm:adaptive-offline-deterministic}
Niech $\ALG$ będzie zrandomizowanym algorytmem, który jest
\mbox{$\gamma$-konkurencyjny} przeciwko adwersarzowi adaptive-offline. Wtedy
istnieje deterministyczny $\gamma$-kon\-ku\-ren\-cyj\-ny algorytm.
\end{theorem}

Najpierw przyjrzymy się podejściu które nie działa.  Ponownie traktujemy $\ALG$
jako rozkład prawdopodobieństwa nad wszystkimi algorytmami deterministycznymi
$\DET_i$.  Z konkurencyjności algorytmu $\ALG$ wynika, że dla każdego
adwersarza adaptive-offline $(Q,\OPT)$ dla dowolnego wejścia zachodzi
$\E_i[\DET_i(\sigma(\DET_i,Q))] \leq \gamma(\E_i[\OPT(\sigma(\DET_i,Q))])$.
Zatem 
\[
	\E_i[\,\DET_i(\sigma(\DET_i,Q)) - \gamma(\OPT(\sigma(\DET_i,Q))) ] \leq 0 \enspace.
\]
Stąd wynika, że istnieje algorytm $\DET_i$ taki, że
\[
	\DET_i(\sigma(\DET_i,Q)) \leq \gamma(\OPT(\sigma(\DET_i,Q))) \enspace.
\]
Niestety algorytm taki istnieje dla ustalonego $Q$; my potrzebujemy, żeby
istniał jeden $\DET_j$ dobry dla wszystkich~$Q$.  Poniżej prezentujemy inne
podejście do twierdzenia, tym razem skuteczne.

\marginnote{W.13B}

\begin{proof}[Dowód twierdzenia \ref{thm:adaptive-offline-deterministic}]
Rozważmy drzewo gry pomiędzy adwersarzem a algorytmem. Zakładamy, że korzeń
jest na poziomie $1$ i reprezentuje pustą sekwencję wejściową. Krawędzie
wychodzące z poziomów nieparzystych reprezentują możliwe zapytania, a krawędzie
wychodzące z poziomów parzystych możliwe odpowiedzi algorytmu.

W tej terminologii $Q$ jest po prostu funkcją, która dla każdego wierzchołka na
poziomie nieparzystym wybiera wychodzącą z niego krawędź, a $\ALG$ definiuje
dla każdego wierzchołka na poziomie parzystym rozkład prawdopodobieństwa nad
wszystkimi wychodzącymi z niego krawędziami.  Każdy wierzchołek jest
identyfikowany przez parę $(\sigma, a)$, gdzie $\sigma$ jest wygenerowanym
początkiem sekwencji wejściowej a $a$ jest odpowiedzią algorytmu. 
Z wierzchołkiem takim wiążemy wartość $cost(\sigma,a)$, która jest 
równa kosztowi rozwiązania $a$ na sekwencji $\sigma$.

Należący do nieparzystego poziomu wierzchołek $(\sigma,a)$ nazywamy {\em
natychmiast wygrywającym} (dla adwersarza) jeśli zachodzi dla niego
$cost(\sigma,a) > \gamma (\OPT(\sigma))$. Jeśli algorytm dojdzie do takiego
wierzchołka, adwersarz natychmiast przerywa grę. Należący do nieparzystego
poziomu wierzchołek $(\sigma,a)$ nazywamy {\em wygrywającym} jeśli istnieje
taka stała $B$, że przy optymalnej grze adwersarza każda ścieżka, którą może
wybrać algorytm zawiera natychmiast wygrywający wierzchołek w ciągu co najwyżej
$B$ kroków.  Powyższe definicje wierzchołków wygrywających i natychmiast
wygrywających zależą tylko od struktury drzewa (i funkcji kosztu) a nie od
konkretnych wartości $Q$ i $\ALG$.

Zauważmy, że korzeń drzewa nie może być wierzchołkiem wygrywającym, bo
oznaczałoby to, że niezależnie od działań algorytmu adwersarz może wygenerować
sekwencję, nie dłuższą niż~$B$, która spowoduje, że algorytm skończy w
wierzchołku natychmiast wygrywającym. Wtedy skonstruowanie
$\gamma$-konkurencyjnego algorytmu byłoby niemożliwe, co przeczy założeniu. 

Pokażemy teraz, że dla dowolnego niewygrywającego wierzchołka $x$ dowolne
żądanie adwersarza może być obsłużone (deterministycznie) tak, że algorytm skończy nadal w
wierzchołku niewygrywającym. Indukcyjnie będzie to prowadzić do
deterministycznego algorytmu, który będzie się poruszać zawsze po wierzchołkach
niewygrywających, a zatem będzie $\gamma$-konkurencyjny. Załóżmy nie wprost, że
istnieje syn wierzchołka $x$, $y$, taki że każdy syn wierzchołka $y$ jest
wygrywający. Ponieważ dla każdego syna $y$ istnieje zatem ograniczenie na
liczbę kroków która doprowadza do wierzchołka natychmiast wygrywającego i synów
jest skończona ilość, istnieje też ograniczenie na liczbę kroków która prowadzi
z wierzchołka $x$ do wierzchołka natychmiast wygrywającego.  Zatem $x$ jest
wygrywający, co przeczy założeniu.
\end{proof}

