\section{Routing bez primal-dual}

Niech $m = |E|$
Skonstruujemy algorytm $O(U)$-konkurencyjny, jeśli $U \geq \log 2 m$. Da się go poprawić, żeby był 
$O(\log m)$ konkurencyjny. 

Intuicyjnie chcemy wybierać takie ścieżki, które omijają przeciążone krawędzie. 
Z drugiej strony może to prowadzić do wyboru długich ścieżek. Nieopłacalne ścieżki będziemy odrzucać.
Chcemy zatem optymalizować miarę $F_j = \sum_{e \in E} 2^{\ell_j(e)} - 1$.

Nasz algorytm będzie sprawdzać w kroku $j$ czy istnieje ścieżka $P$ od $s_j$ do $t_j$, dla której
zachodzi
\[
	\sum_{e \in P} 2^{\ell_{j-1}(e)} \leq m
	\enspace,
\]
W takim przypadku akceptuje połączenie wybierając powyższą ścieżkę, inaczej połączenie jest odrzucane. Zauważmy, że
algorytm zaakceptuje pierwszą podaną ścieżkę. Poniżej $T$ oznacza długość wejścia.

\begin{lemma}
Rozwiązanie generowane przez algorytm jest dopuszczalne.
\end{lemma}

\begin{proof}
Ustalmy krawędź $e$. 
Niech $P_k$ będzie ostatnią ścieżkę, która przechodzi przez $e$. 
Ponieważ ścieżka ta została zaakceptowana zachodzi
\[
	\sum_{e \in P_k} 2^{\ell_{k-1}(e)} \leq m
	\enspace,
\]
a zatem w szczególności $\ell_{k-1}(e) \leq \log m$ i stąd $\ell_k(e) = \ell_{k-1}(e)+1 \leq \log 2 m \leq U$.
\end{proof}

\medskip

Zauważmy, że na początku $F_0 = 0$.
Porównamy zysk \ALG i \OPT do~$F_T$.

\begin{lemma}
Niech $A$ będzie zbiorem połączeń zaakceptowanych przez \ALG. Wtedy 
$m \cdot |A| \geq F_T$.
\end{lemma}


\begin{proof}
Zauważmy, że jeśli \ALG odrzuca połączenie w kroku $j$ to $F_j = F_{j-1}$.
Natomiast jeśli \ALG akceptuje połaczenie, to 
\begin{align*}
	F_j - F_{j-1} 
	= \sum_{e \in P_j} \left( 2^{\ell_{j}(e)} - 2^{\ell_{j-1}(e)} \right) 
	= \sum_{e \in P_j} 2^{\ell_{j-1}(e)} \leq m
	\enspace.
\end{align*}
W ostatniej nierówności skorzystaliśmy z definicji algorytmu.
\end{proof}


\begin{lemma}
Niech $B$ będzie zbiorem połączeń odrzuconych przez \ALG ale zaakceptowanych przez \OPT.
Wtedy $m \cdot |B| \leq F_T \cdot U + U \cdot m$.
\end{lemma}%
\begin{proof}
Weźmy dowolne połączenie $j \in B$, niech $P^*_j$ będzie ścieżką użytą przez \OPT. 
Ponieważ \ALG odrzucił to żądanie, zachodzi
\[
	m < \sum_{e \in P^*_j} 2^{\ell_{j-1}(e)} \leq 
	\sum_{e \in P^*_j} 2^{\ell_{T}(e)}
	\enspace.
\]
Wysumujmy teraz powyższą nierówność po wszystkich $j \in B$. Ponieważ
\OPT jest rozwiązaniem dopuszczalnym, każda krawędź pojawi się w sumie co najwyżej $U$ razy:
\begin{align*}
	m \cdot |B| < 
	\sum_{j \in B} \sum_{e \in P^*_j} 2^{\ell_{T}(e)}
	& = \sum_{e \in E} \sum_{j: e \in P^*_j} 2^{\ell_{T}(e)} 
	\leq \sum_{e \in E} U \cdot 2^{\ell_{T}(e)} \\
	& = \sum_{e \in E} U \cdot (2^{\ell_{T}(e)} - 1) + \sum_{e \in E} U \cdot 1 \\
	& = U \cdot F_T + U \cdot m \enspace.
	\qedhere
\end{align*}
\end{proof}

\begin{theorem}
Jeśli Algorytm jest ściśle $O(U)$-konkurencyjny.
\end{theorem}

\begin{proof}
Używamy $|A| \geq 1$. Mamy
$\OPT \leq |B| + |A| \leq U \cdot (F_t/m) + U + |A| \leq U \cdot |A| + U \cdot |A| + |A| = (2U+1) \cdot |A| 
= (2U+1) \cdot \ALG$. 
\end{proof}

Dla $U \geq \log 2d$ można optymalizować $F_j = \sum_{e \in E} (4m)^{\ell_{j}(e) / U} - 1$ i cały dowód będzie 
prawie identyczny, ale na końcu dostaniemy $O(\log m)$-konkurejncyjność.


%Niech 
%\[
%	c_j(e) = (4d)^{\ell_{j}(e) / U} 
%\]
%(Zauważmy, że dla $U = \log 4d$ zachodzi $c_{j-1}(e) = 2^{\ell_{j-1}(e)}$).
%Nasz algorytm będzie sprawdzać w kroku $j$ czy istnieje ścieżka $P$ od $s_j$ do $t_j$, dla której
%zachodzi
%\[
%	\sum_{e \in P} c_{j-1}(e) \leq 2 d  
%	\enspace,
%\]
%W takim przypadku akceptuje połączenie wybierając powyższą ścieżkę, inaczej połączenie jest odrzucane.
%Poniżej $m$ oznacza długość wejścia.%

%\begin{lemma}
%Rozwiązanie generowane przez algorytm jest dopuszczalne.
%\end{lemma}%

%\begin{proof}
%Ustalmy krawędź $e$.
%Niech $P_k$ będzie ostatnią ścieżkę, która przechodzi przez $e$. Pokażemy, ze $\ell_k(e) \leq U$.
%Ponieważ żądanie $k$ zostało zaakceptowane zachodzi	$\sum_{e \in P_k} c_{k-1}(e) \leq 2 d$
%a zatem w szczególności $(4d)^{\ell_{k-1}(e) / U}  = c_{k-1}(e) \leq 2 d$.
%	
%A zatem bezpośrednio $\ell_{k-1}(e) / U < 1$. To jeszcze nie wystarcza, ale
%\[
%	(4d)^{\ell_{k}(e) / U} = (4d)^{\ell_{k-1}(e) / U} \cdot (4d)^{1/U} \leq 2d \cdot (4d)^{1/U}
%	\leq 2d \cdot (4d)^{1/(4d)} = 4d
%	\enspace,
%\]
%a zatem $\ell_{k}(e) \leq U$.
%\end{proof}%

%Będziemy teraz porównywać wielkość $\sum_e c_m(e)$ do zysku \ALG i \OPT.
%Wartość $X_j(m) \sum_e c_m(e)$%

%\begin{lemma}
%Jeśli algorytm akceptuje połączenie %
%

%Niech $A$ będzie zbiorem zaakceptowanych połączeń.
%Po wykonaniu algorytmu zachodzi $U \cdot \sum_e c_m(e) \leq 4 d \log 4 d \cdot |A|$.
%\end{lemma}%
%

%\begin{lemma}
%Niech $A$ będzie zbiorem zaakceptowanych połączeń.
%Po wykonaniu algorytmu zachodzi $U \cdot \sum_e c_m(e) \leq 4 d \log 4 d \cdot |A|$.
%\end{lemma}%

%\begin{proof}
%Pokażemy, że po zaakceptowaniu połączenia w kroku $j$ 
%zachodzi
%\begin{equation}
%	U \cdot \sum_{e \in P_j} (c_j(e) - c_{j-1}(e)) \leq 4 d \log 4d
%	\enspace.
%\end{equation} 
%Ta nierówność wysumowana po wszystkich akceptujących krokach da tezę lematu.%

%Ustalmy dowolną krawędź $e \in P_j$. 
%Wtedy 
%\begin{align*}
%c_j(e) - c_{j-1}(e)
%	\leq &\; (4d)^{\ell_{j}(e) / U} - (4d)^{\ell_{j-1}(e) / U} \\
%	= &\; (4d)^{\ell_{j-1}(e) / U} \cdot \left( (4d)^{1 / U} - 1 \right) \\
%	= &\; (4d)^{\ell_{j-1}(e) / U} \cdot \left( 2^{\log (4d) / U} - 1 \right) \\
%	\leq &\; (4d)^{\ell_{j-1}(e) / U} \cdot 2 \cdot \log (4d) / U
%\end{align*}
%(Druga nierówność wynika z $2^x - 1 \leq 2x$ dla dowolnego $x \in [0,1]$).
%Sumując to po wszystkich krawędziach z $P_j$ otrzymujemy
%\begin{align*}
%	U \cdot \sum_{e \in P_j} (c_j(e) - c_{j-1}(e)) 
%	\leq &\; 2 \cdot \log 4d \cdot \sum_{e \in P_j} (4d)^{\ell_{j-1}(e) / U}  \\
%	\leq &\; 2 \cdot \log 4d \cdot 2 d 
%	\enspace.
%\end{align*}
%W ostatniej nierówności skorzystaliśmy z definicji algorytmu.
%\end{proof}%
%

%\begin{lemma}
%Niech $B_m$ będą indeksami tych połączeń, które są odrzucone przez \ALG ale zaakceptowane przez \OPT.
%Wtedy $d \cdot |B_m| \leq U \cdot \sum_e c_m(e)$.
%\end{lemma}%

%\begin{proof}
%Dla każdego połączenia $j$ odrzuconego przez \ALG i puszczonego przez \OPT ścieżką $P^*_j$ zachodzi 
%\[
%	2 d < \sum_{e \in P^*_j} c_{j-1}(e) \leq \sum_{e \in P^*_j} c_m(e)
%	\enspace.
%\]
%Wysumujmy teraz powyższą nierówność po wszystkich takich ścieżkach $P^*j$. Ponieważ
%\OPT jest rozwiązaniem dopuszczalnym dla każdej krawędzi $e$ wyrażenie $c_m(e)$
%pojawi się po prawej stronie co najwyżej $U$ razy.
%\[
%	2 d \cdot |B_m| < \sum_e U \cdot c_m(e) 
%	\enspace.
%	\qedhere
%\]
%\end{proof}%

%\begin{theorem}
%Algorytm jest $O(\log d)$-konkurencyjny.
%\end{theorem}%

%\begin{proof}
%Niech $C = U \cdot \sum_e c_m(e) / d$. Wtedy
%\[
%	\OPT = |A_m| + |B_m| \leq C + |A_m| \leq (4 \log d + 1) \cdot |A_m| = (4 \log d + 1) \cdot \ALG 
%\enspace.
%\qedhere
%\]
%\end{proof}



