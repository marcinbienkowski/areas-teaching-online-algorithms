
\section{Maksymalne skojarzenie online}

\problem{Maksymalne skojarzenie}{
Mamy dane dwa równoliczne zbiory dziewczyn $D$ i chłopców $C$, oba o mocy $n$. 
Wiemy, że graf połączeń między $C$ i $D$ jest dwudzielny, lecz jest nieznany na początku;
połączenie oznacza, że dane dwie osoby o różnej płci się znają.

$\enspace\enspace$ 
Wejście składa się z $n$ kroków; w jednym kroku $1 \leq i \leq n$ poznajemy wszystkie krawędzie prowadzące 
od chłopca $c_i$ do znanych mu dziewczyn. Należy wybrać jedną niewybraną jeszcze dziewczynę z którą $c_i$ 
jest połączony; tę parę nazywamy skojarzoną. Należy zmaksymalizować moc zbioru skojarzeń. 
}

Rozważmy algorytm zachłanny, tj.~taki, który wybiera do skojarzenia dowolną krawędź prowadzącą od
danego chłopca do jeszcze wolnej dziewczyny.

\begin{theorem}
Przy założeniu, że w grafie istnieje doskonałe 
skojarzenie (o mocy $n$), powyższa zachłanna strategia jest $2$-konkurencyjna.
\end{theorem}

\begin{proof}
Zauważmy, że algorytm zachłanny zwraca skojarzenie, które jest maksymalne, tj.~nie daje się rozszerzyć.
Dla instancji $I$ dopuszczającej doskonałe skojarzenie $\OPT(I) = n$.
Zatem wystarczy pokazać, że $\ALG(I) \geq n/2$. 

Załóżmy, że tak nie jest. Oznacza to, że podczas działania 
algorytmu podzbiór chłopców $C'$ został połączony ze zbiorem dziewczyn $D'$, $|C'| = |D'| < n/2$
a pozostali chłopcy (ze zbioru $C \setminus C'$) nie zostali połączeni.
Maksymalność skojarzenia implikuje, że w grafie mogą oni sąsiadować wyłącznie z 
już zajętymi dziewczynami (ze zbioru $D'$).

Z drugiej strony $|C \setminus C'| > n/2$, a zatem z twierdzenia Halla\footnote{Jak założyliśmy, 
w grafie istnieje skojarzenie doskonałe.} wynika, że zbiór sąsiadów tego zbioru musi mieć
co najmniej tyle samo elementów. Otrzymana sprzeczność kończy dowód.
\end{proof}

%%%%%%%%%%%%%%%%%%%%%%%%%%%%%%%%%%%%%%

