\documentclass[a4paper,11pt]{article}
\usepackage[margin=2cm,bottom=2cm]{geometry}
\usepackage{amsmath,amsthm,amsopn,amsfonts}
\usepackage{polski}
\usepackage[utf8]{inputenc}
\usepackage{enumerate}
\usepackage{graphicx}
\pagestyle{empty}

\newcommand{\makeheader}[1]{
\begin{center}
{\Large\bf\sectfont Algorytmy online\\}
\vspace{0.3cm}
{\large\bf\sectfont Lista #1}\\
\end{center}
\vspace{0.7cm}
}


\newcommand{\myfigure}[3]{
\begin{figure}[t]
\centering
\fboxsep7pt
\framebox[0.95\textwidth]{
    \begin{minipage}{0.95\textwidth}
        \centering #3
    \end{minipage}
}
\caption{#2}
\label{fig:#1}
\end{figure}
}



\newcommand{\makefooter}{
\bigskip
\hfill {\em Marcin Bieńkowski}
}

\newcommand{\E}{\mathbf{E}}
\newcommand{\ALG}{\textsc{Alg}}
\newcommand{\OPT}{\textsc{Opt}}
\newcommand{\e}{\textnormal{e}}
\newcommand{\CNT}{\textsc{Cnt}}


\begin{document}
\makeheader{3}


\begin{description}

\item[Zadanie 1.] \textbf{(3 pkt.)} 
Rozważmy wariant problemu MTS (\emph{Metrical Task System}) z metryką dyskretną 
\mbox{$n$-elementową}, gdzie odległość między każdymi dwoma różnymi punktami wynosi $D \in \mathbb{N}$.
W każdym kroku wektor kar zawiera tylko zera i jedynki. Skonstruuj
$O(\log n)$-konkurencyjny randomizowany algorytm. Wskazówka: zrandomizuj  
deterministyczy algorytm zaznaczający z~wykładu.

\item[Zadanie 2.] \textbf{(2 pkt.)}
Rozważmy następujące algorytmy dla problemu reorganizacji listy długości
$\ell$.  Algorytm \textsc{Transpose} po odwołaniu do elementu $x$ przesuwa $x$
o jedno miejsce w stronę początku listy.  Algorytm \textsc{Frequency Count} po
wystąpieniu żądania reorganizuje listę, tak żeby była posortowana pod względem
częstotliwości wystąpienia elementów w już widzianej części sekwencji
wejściowej.  Pokaż, że konkurencyjność obu tych algorytmów
wynosi~$\Omega(\ell)$.  
%Wskazówka: co się dzieje jeśli adwersarz pyta zawsze o ostatni elementy listy?
%Wskazówka: rozważ sekwencję $\sigma = x_1^k, x_2^{k-1}, x_3^{k-2}, \ldots, x_\ell^{k+1-\ell}$.

\item[Zadanie 3.] \textbf{(3 pkt.)} 
Wykorzystując funkcje potencjału, udowodnij, że dla listy dwuelementowej algorytm
\textsc{Move To Front} jest $4/3$-konkurencyjny. 
Funkcja potencjału nie powinna zależeć od historii, tylko 
od bieżącego stanu listy \textsc{Opt} i \textsc{Mtf}. 

\item[Zadanie 4.] \textbf{(2 pkt.)} 
Wykorzystując funkcje potencjału, udowodnij, że algorytm, który kupuje narty
dnia $B$, jest $2$-konkurencyjny.  Funkcja potencjału może zależeć od stanu
algorytmu, stanu \textsc{Opt} i numeru dnia.

%\item[Zadanie 5] \textbf{(1 pkt.)}
%Pokaż sekwencję wejściową dla problemu reorganizacji listy, dla której rozwiązanie optymalne
%musi wykorzystywać płatne zamiany.

% jeśli OPT nie ma nart, to:
% phi_i = i w dni i<B, 0 w dniu i>=B
% jeśli OPT ma narty, to:
% phi_i = 2B-i w dniu i<B, 0 w dniu i>=B
%
% pokazujemy to dla dwoch akcji: 
% 1. OPT może kupić narty
% 2. ALG i OPT placa za wypozyczanie i ALG byc moze kupuje.


\end{description}

\makefooter
\end{document}

