\documentclass[a4paper,11pt]{article}
\usepackage[margin=2cm,bottom=2cm]{geometry}
\usepackage{amsmath,amsthm,amsopn,amsfonts}
\usepackage{polski}
\usepackage[utf8]{inputenc}
\usepackage{enumerate}
\usepackage{graphicx}
\pagestyle{empty}

\newcommand{\makeheader}[1]{
\begin{center}
{\Large\bf\sectfont Algorytmy online\\}
\vspace{0.3cm}
{\large\bf\sectfont Lista #1}\\
\end{center}
\vspace{0.7cm}
}


\newcommand{\myfigure}[3]{
\begin{figure}[t]
\centering
\fboxsep7pt
\framebox[0.95\textwidth]{
    \begin{minipage}{0.95\textwidth}
        \centering #3
    \end{minipage}
}
\caption{#2}
\label{fig:#1}
\end{figure}
}



\newcommand{\makefooter}{
\bigskip
\hfill {\em Marcin Bieńkowski}
}

\newcommand{\E}{\mathbf{E}}
\newcommand{\ALG}{\textsc{Alg}}
\newcommand{\OPT}{\textsc{Opt}}
\newcommand{\e}{\textnormal{e}}
\newcommand{\CNT}{\textsc{Cnt}}


\begin{document}
\makeheader{11}


\begin{description}

\item[Zadanie 1.]  \textbf{(2 pkt.)}
Przypomnijmy, że algorytm \textsc{Move-To-Min} (MTM) dla problemu migracji pliku. 
na końcu fazy przenosił plik do wierzchołka $\arg \min_x \sum_{i=1}^D d(x, r_i)$.
Algorytm \textsc{Move-To-Local-Min} (MTLM) jest jego modyfikacją, która przenosi 
plik do wierzchołka $\arg \min_x D \cdot d(v_\textsc{ALG}, x) + 2 \cdot \sum_{i=1}^D d(x, r_i)$.
W powyższym wzorze $v_\textsc{ALG}$ jest wierzchołkiem, w którym algorytm ma plik w~trakcie fazy.
Pokaż, że konkurencyjność algorytmu MTLM wynosi co najmniej $5$.\footnote{Możesz założyć,
że $D$ jest wielokrotnością wybranej przez Ciebie liczby.}

\emph{Wskazówka:} najprościej wykorzystać do tego celu program liniowy z poprzedniej listy. 
Wystarczy dopisać w nim wymaganie \texttt{potential\_start = potential\_end;}
i odpowiednio zmienić 3 wiersze definiujące algorytm.

% A0 -----------4------------ O1=O2 -----------3------- R ---1----A1


\item[Zadanie 2.]  
W tym i następnym zadaniu rozważamy problem migracji pliku na grafie o dwóch
wierzchołkach $v_0$ i $v_1$ połączonych krawędzią długości $1$, gdzie rozmiar pliku jest równy 
$D \in \mathbb{N}$, zaś algorytm bezpośrednio po zobaczeniu żądania w kroku $t$ i zapłaceniu za nie 
oblicza funkcję pracy~$w_t$. 

\begin{enumerate}
\item 
\textbf{(1 pkt.)}
Niech $W_t = (w_t(v_0) + w_t(v_1)) / 2$.
Udowodnij, że dla dowolnego wejścia $\sigma$ o długości~$T$ zachodzi $W_T \leq \OPT(\sigma) + O(D)$.

\item 
\textbf{(3 pkt.)}
Niech $v_i$ będzie wierzchołkiem, w którym algorytm \DET ma obecnie plik.
	Jeśli $w_t(v_i) \geq w_t(v_{1-i}) + D$, to \DET przenosi plik do $v_{1-i}$. Pokaż, 
	że \DET jest $O(1)$-konkurencyjny. Jakie najlepsze oszacowanie potrafisz uzyskać?
\end{enumerate}


\item[Zadanie 3.] \textbf{(4 pkt.)} 
	Rozważmy model, w którym \OPT musi przechowywać plik w jednym z wierzchołków,
	ale algorytm online może przechowywać plik w dowolnym miejscu krawędzi pomiędzy 
	wierzchołkami $v_0$ i $v_1$. 
	Zakładamy że $v_0$ ma współrzędną $0$ zaś $v_1$ współrzędną $1$.
	Po obliczeniu funkcji pracy $w_t$, algorytm \CONT przenosi plik 
	do punktu o współrzędnej
	\[
		p_t = \frac{1}{2} + \frac{w_t(v_0) - w_t(v_1)}{2 D} \,.
	\]
	Pokaż, że \CONT jest $O(1)$-konkurencyjny. Jakie najlepsze oszacowanie potrafisz uzyskać?

\end{description}

\makefooter
\end{document}

