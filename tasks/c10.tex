\documentclass[a4paper,11pt]{article}
\usepackage[margin=2cm,bottom=2cm]{geometry}
\usepackage{amsmath,amsthm,amsopn,amsfonts}
\usepackage{polski}
\usepackage[utf8]{inputenc}
\usepackage{enumerate}
\usepackage{graphicx}
\pagestyle{empty}

\newcommand{\makeheader}[1]{
\begin{center}
{\Large\bf\sectfont Algorytmy online\\}
\vspace{0.3cm}
{\large\bf\sectfont Lista #1}\\
\end{center}
\vspace{0.7cm}
}


\newcommand{\myfigure}[3]{
\begin{figure}[t]
\centering
\fboxsep7pt
\framebox[0.95\textwidth]{
    \begin{minipage}{0.95\textwidth}
        \centering #3
    \end{minipage}
}
\caption{#2}
\label{fig:#1}
\end{figure}
}



\newcommand{\makefooter}{
\bigskip
\hfill {\em Marcin Bieńkowski}
}

\newcommand{\E}{\mathbf{E}}
\newcommand{\ALG}{\textsc{Alg}}
\newcommand{\OPT}{\textsc{Opt}}
\newcommand{\e}{\textnormal{e}}
\newcommand{\CNT}{\textsc{Cnt}}


\begin{document}
\makeheader{10}


\begin{description}

\item[Zadanie 1.] \textbf{(2 pkt.)}
	Pokaż, że konkurencyjność dowolnego algorytmu deterministycznego \DET dla
	problemu przenoszenia pliku wynosi co najmniej $3$. Wskazówka: rozważ graf
	dwuwierzchołkowy i porównaj \DET z trzema algorytmami (dwa statyczne i
	jeden, który ma plik w innym wierzchołku niż \DET).
	
\item[Zadanie 2.] \textbf{(2 pkt.)}
	Rozważ następujący algorytm $\CNT_k$ parametryzowany pewną stałą $k$. Na
	początku działania $\CNT_k$ losuje wartość licznika $C$ jednostajnie z przedziału
	$[1,k]$. W każdym kroku, po obsłudze żądania, $\CNT_k$ zmniejsza
	$C$ o $1$ i jeśli $C = 0$, to plik jest przenoszony do
	miejsca żądania, zaś $C$ jest zmieniane na $k$.

	Pokaż, że dla dowolnego $k$ konkurencyjność algorytmu $\CNT_k$
	wynosi co najwyżej 
	\[ 
		\max \left \{2 + \frac{2D}{k}, 1 + \frac{k}{2D} \right \} + O(1/D) 
		\enspace.
	\] 
	gdzie $D$ jest rozmiarem pliku. 
	Możesz wykorzystać funkcję 
	potencjału równą $\Phi = (C+D) \cdot d(v_\textnormal{CNT}, v_\textnormal{OPT})$. 
	Zakładając, że $D \to \infty$ i nie musisz się przejmować całkowitością $k$, oblicz jakie jest 
	$k$ optymalizujące powyższe maksimum i ile wynosi wtedy konkurencyjność $\CNT_k$.
	


\item[Zadanie 3.] \textbf{(2 pkt.)}
Rozważmy następujące uogólnienie algorytmu \textsc{Flip} dla problemu przenoszenia pliku: 
algorytm \textsc{Flip}($p$) przenosi plik do żądania z prawdopodobieństwem $p \in (0,1]$. (Oryginalny 
algorytm \textsc{Flip} wybierał $p = 1/(2D)$). 
Ustalmy graf, który składa się z dwóch wierzchołków połączonych krawędzią. 
Pokaż, że dla dowolnego $p$ 
konkurencyjność algorytmu \textsc{Flip}($p$) 
na tym grafie wynosi co najmniej $3$.
% Algorytm siedzi w v1. Jeśli p <= 1/(2D), to dajemy zadanie (2^n 1^n) ^ \infty
% Jeśli p >= 1/(2D), to dajemy zadanie (2 1^n)^ \infty


\item[Zadanie 4.]
Na odwrocie znajdziesz program liniowy, zapisz go do pliku \texttt{lower.lp}.
Przejrzyj jego zawartość zwracając uwagę na trzy wyróżnione nierówności:
dlaczego są one prawdziwe? 
Uruchom polecenie 
\texttt{lp\_solve < lower.lp}.\footnote{Program 
\texttt{lp\_solve} to darmowy solver LP dostępny w większości popularnych
systemów operacyjnych. Jeśli w wyniku otrzymasz inne wartości niż
\texttt{potential\_start = 0} i \texttt{potential\_end = 4}, to wymuś je
odkomentowując odpowiednie wiersze.}  

\begin{enumerate}

\item \textbf{(2 pkt.)} Na podstawie wyniku narysuj i opisz, jak wygląda faza
dla której zachodzi $\ALG + \Delta \Phi = 7 \cdot \OPT$, gdzie $\Phi = 2 \cdot D
\cdot d(\ALG, \OPT)$. Uwaga: prawdopodobnie żądania nie mogą być skupione w
jednym miejscu \texttt{req}, ten punkt odpowiada tylko ich ,,środkowi
masy''.

% A_0,O_0,O_1,polowa req ----------------dlugosc-2--------------polowa req,A_1


\item \textbf{(2 pkt.)} Zauważ, że powyższy wynik nie prowadzi łatwo do
prawdziwego dolnego ograniczenia, bo odległość $d(\ALG, \OPT)$ jest inna na
początku i na końcu fazy. Dodaj odpowiednią równość, żeby to wymusić, wykonaj
ponownie program i rozwiąż ponownie poprzedni podpunkt. Jak z obecnej
konstrukcji wynika dolne ograniczenie $7$ na konkurencyjność algorytmu
\texttt{Move-To-Min}?


% A_0 --------dlugosc-2---------- O_0,O_1,polowa req -----------dlugosc-2---------polowa req,A_1


\end{enumerate}

\end{description}

\makefooter

\newpage

\begin{Verbatim}[fontsize=\footnotesize]
/* 
  Generowanie metryki wymuszającej największy współczynnik dla jednej fazy algorytmu Move-To-Min 
  dla problemu File migration i konkretnej funkcji potencjału równej 2 * D * dist(ALG, OPT).

  Mamy dane punkty alg0, alg1, opt0 i opt1 (pozycje ALG i OPT na początku i końcu). Zmienne w programie 
  to głównie D_x_y, oznaczające odległość między punktami x i y, pomnożoną przez D. Dodatkowo 
  D_req_x oznacza sumę odległości od punktu x do wszystkich żądań z danej fazy. 

  UWAGA: Wartość ratio jest GÓRNYM ograniczeniem, tj. jeśli program zwróci ratio = 7, to oznacza, 
  że faktyczna konkurencyjność to co najwyżej 7. Ale nie na odwrót: może wygenerować układ punktów, 
  który NIE JEST MOŻLIWY DO ZAIMPLEMENTOWANIA jako prawdziwa faza. 
*/

max: ratio;

ratio = alg_cost + potential_end - potential_start;

alg_cost = D_req_alg0 + D_alg0_alg1;
opt_cost = 1;

opt_cost >= D_req_opt0;		// Lemat ograniczający koszt OPT zastosowany do opt0
opt_cost >= D_req_opt1;		// Lemat ograniczający koszt OPT zastosowany do opt1
opt_cost >= D_opt0_opt1;

potential_start = 2 D_alg0_opt0;
potential_end   = 2 D_alg1_opt1;

// Odkomentuj, jeśli masz za dużo szczęścia:
// potential_start = 0;
// potential_end = 4;

// Jakie jest znaczenie tych trzech nierówności?
D_req_alg1 <= D_req_alg0;
D_req_alg1 <= D_req_opt0;
D_req_alg1 <= D_req_opt1;

// Nierówności trójkąta.
D_req_alg0  <= D_req_alg1  + D_alg0_alg1;
D_req_alg0  <= D_req_opt0  + D_alg0_opt0;
D_req_alg0  <= D_req_opt1  + D_alg0_opt1;
D_req_alg1  <= D_req_alg0  + D_alg0_alg1;
D_req_alg1  <= D_req_opt0  + D_alg1_opt0;
D_req_alg1  <= D_req_opt1  + D_alg1_opt1;
D_req_opt0  <= D_req_alg0  + D_alg0_opt0;
D_req_opt0  <= D_req_alg1  + D_alg1_opt0;
D_req_opt0  <= D_req_opt1  + D_opt0_opt1;
D_req_opt1  <= D_req_alg0  + D_alg0_opt1;
D_req_opt1  <= D_req_alg1  + D_alg1_opt1;
D_req_opt1  <= D_req_opt0  + D_opt0_opt1;
D_alg0_alg1 <= D_req_alg0  + D_req_alg1;
D_alg0_alg1 <= D_alg0_opt0 + D_alg1_opt0;
D_alg0_alg1 <= D_alg0_opt1 + D_alg1_opt1;
D_alg0_opt0 <= D_req_alg0  + D_req_opt0;
D_alg0_opt0 <= D_alg0_alg1 + D_alg1_opt0;
D_alg0_opt0 <= D_alg0_opt1 + D_opt0_opt1;
D_alg0_opt1 <= D_req_alg0  + D_req_opt1;
D_alg0_opt1 <= D_alg0_alg1 + D_alg1_opt1;
D_alg0_opt1 <= D_alg0_opt0 + D_opt0_opt1;
D_alg1_opt0 <= D_req_alg1  + D_req_opt0;
D_alg1_opt0 <= D_alg0_alg1 + D_alg0_opt0;
D_alg1_opt0 <= D_alg1_opt1 + D_opt0_opt1;
D_alg1_opt1 <= D_req_alg1  + D_req_opt1;
D_alg1_opt1 <= D_alg0_alg1 + D_alg0_opt1;
D_alg1_opt1 <= D_alg1_opt0 + D_opt0_opt1;
D_opt0_opt1 <= D_req_opt0  + D_req_opt1;
D_opt0_opt1 <= D_alg0_opt0 + D_alg0_opt1;
D_opt0_opt1 <= D_alg1_opt0 + D_alg1_opt1;
\end{Verbatim}

 
%W szczególności:
%- Ograniczenie na OPT może nie być ciasne.
%- Funkcja potencjału może nie być dobrze dobrana.
%- W momencie konstrukcji inputu musimy podjąć decyzję gdzie rozmieścić żądania, żeby zmienne D_req_x odpowiadały sumie %odległości 
%  do żądań. Może to wygenerować dodatkowe punkty w grafie, które będą dla algorytmu bardziej atrakcyjne niż alg1.
%W dodatku generowana jest jedna faza: nie mamy automatycznie gwarancji, że będzie z niej złożyć dowolnie długie %wejście dowodzące 
%dolnego ograniczenia.

\end{document}
