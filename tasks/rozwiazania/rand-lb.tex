\documentclass[a4paper,11pt]{scrartcl}
\usepackage[margin=2cm,bottom=2cm]{geometry}
\usepackage{amsmath,amsthm,amsopn,amsfonts}
\usepackage{polski}
\usepackage[utf8]{inputenc}
\usepackage{enumerate}
\usepackage{graphicx}
\pagestyle{empty}

\newcommand{\makeheader}[1]{
\begin{center}
{\Large\bf\sectfont Algorytmy online\\}
\vspace{0.3cm}
{\large\bf\sectfont Lista #1}\\
\end{center}
\vspace{0.7cm}
}


\newcommand{\myfigure}[3]{
\begin{figure}[t]
\centering
\fboxsep7pt
\framebox[0.95\textwidth]{
    \begin{minipage}{0.95\textwidth}
        \centering #3
    \end{minipage}
}
\caption{#2}
\label{fig:#1}
\end{figure}
}



\newcommand{\makefooter}{
\bigskip
\hfill {\em Marcin Bieńkowski}
}

\newcommand{\E}{\mathbf{E}}
\newcommand{\ALG}{\textsc{Alg}}
\newcommand{\OPT}{\textsc{Opt}}
\newcommand{\e}{\textnormal{e}}
\newcommand{\CNT}{\textsc{Cnt}}


\begin{document}
\makeheader{4}

\newtheorem{theorem}{Theorem}
\newcommand{\RAND}{{\sc Rand}}



\begin{theorem}
Współczynnik konkurencyjności algorytmu $\RAND$ przeciwko adwersarzowi nieświadomemu wynosi co najmniej $k$.
\end{theorem}

\begin{proof}
Oznaczmy strony, które są na początku w pamięci podręcznej przez 
$a_1, a_2, \ldots, a_k$. Rozważmy następującą sekwencję:
\[
	\sigma = \underbrace{(b_1, a_2, a_3, \ldots, a_k)^\ell}_\textnormal{blok $1$}, 
		\underbrace{(b_2, a_2, a_3, \ldots, a_k)^\ell}_\textnormal{blok $2$}, \ldots,
		\underbrace{(b_\gamma, a_2, a_3, \ldots, a_k)^\ell}_\textnormal{blok $\gamma$} \enspace ,
\]
gdzie $\ell$ zostanie ustalone za chwilę, a wszystkie elementy $b_i$ są różne od elementów $a_j$ (dla 
dowolnych $i,j$ mamy $b_i \neq a_j$). Koszt optymalnego algorytmu w każdym bloku to co najwyżej $1$. 
Pokażemy, że dla dowolnego bloku zachodzi
\begin{equation}
\label{eq:rand_paging_oblivious}
	\E[\RAND(\textsf{blok})] \geq k \cdot \left( 1 - \left( \frac{k-1}{k} \right)^\ell \right)
\end{equation}
Czynnik $(\frac{k-1}{k})^\ell$ zbiega do zera jeśli weźmiemy odpowiednio duże $\ell$; wtedy  
$\E[\RAND(\textsf{blok})]$ jest dowolnie bliskie~$k$. 

Rozważmy zatem dowolny $i$-ty blok, złożony z $\ell$ segmentów $(b_i, a_2, a_3, \ldots, a_k)$. 
Na początku tego bloku $\RAND$ nie ma w swojej pamięci elementu $b_i$, więc pierwsze odwołanie 
w bloku to chybienie.
Mówimy, że chybienie algorytmu {\em kończy się sukcesem}, jeśli $\RAND$ 
wybiera do wyrzucenia taką stronę, że po danym kroku w jego pamięci podręcznej są wszystkie 
ze stron $\{ b_i, a_2, a_3, \ldots, a_k \}$ (od takiej chwili $\RAND$ nie płaci już nic w danym bloku). 
Zauważmy, że chybienie kończy się sukcesem z prawdopodobieństwem nie większym niż $1/k$
(jeśli w momencie chybienia $k-1$ stron jest ,,dobrych'', to jest to dokładnie prawdopodobieństwo, 
że żadnej z nich nie wyrzucimy; w~przeciwnym przypadku nie mamy szans na sukces).

Mamy zatem do czynienia z następującym procesem: jeśli $\RAND$ chybia to płaci
$1$ i z prawdopodobieństwem co najwyżej $1/k$ osiąga sukces (i nie płaci już
więcej). Z pozostałym prawdopodobieństwem za jakiś czas będzie musiał znowu
chybić (najpóźniej wydarzy się to w kolejnym segmencie).  Jeśli zatem
mielibyśmy nieskończenie wiele segmentów ($\ell = \infty$), to liczba chybień
byłaby liczbą kroków do wystąpienia pierwszego sukcesu w procesie niezależnych
prób Bernouliego z prawdopodobieństwem sukcesu $1/k$. Czyli w wartości
oczekiwanej $\RAND$ chybiałby $k$ razy.  Jednak liczba segmentów w bloku jest
jednak skończona i wynosi $\ell$.  Jeśli liczba chybień w bloku wyniesie
powyżej $\ell$ to ,,wybaczamy'' algorytmowi koszt za odwołania powyżej
$\ell$-tego chybienia. Mamy jednak gwarancję, że jeśli algorytmowi przyjdzie
chybiać $\ell$ razy, to te razy zmieszczą się w bloku.  Rozważmy zatem proces
Bernouliego ,,obcięty'' po losowaniu $n$; niech $W_n$ będzie zmienną losową
oznaczającą liczbę kroków w takim procesie. Z powyższych rozważań wynika, że 
$\E[\RAND(\textsf{blok})] \geq \E[W_\ell]$.
Możemy zatem napisać
\begin{align*}
\E[W_\ell] = &\; \ell \cdot \Pr[W_\ell \geq \ell] + \sum_{i=1}^{\ell-1} i \cdot \Pr[W_\ell = i] \\
			= &\; \ell \cdot (1-p)^{\ell-1} + \sum_{i=1}^{\ell-1} i \cdot p \cdot (1-p)^{i-1} \enspace, 
\end{align*}
gdzie $p = 1/k$ i po długich przekształceniach dojść do
(\ref{eq:rand_paging_oblivious}) lub też przyjrzeć się czym jest $\E[W_n]$. Dla
$n = 1$, $W_1 = 1$. Natomiast dla $n \geq 2$ mogą zajść dwa przypadki Albo ---
z prawdopodobieństwem $p$ --- dostaniemy sukces i wtedy $W_n = 1$, albo też
sukcesu nie będzie i wtedy czas oczekiwania jest równy $1 + \E[W_{n-1}]$. Zatem 
\[
	\E[W_n] = p + (1-p) \cdot (1 + \E[W_{n-1}]) = 1 + (1-p) \cdot \E[W_{n-1}] \enspace.
\]
Rozwijając tę zależność $\ell-1$ razy i podstawiając $\E[W_1] = 1$ otrzymujemy
\[
	\E[W_\ell] = 1 + (1-p) + (1-p)^2 + \ldots + (1-p)^{\ell-1} 
	=  \frac{1-(1-p)^\ell}{1-(1-p)} = k \cdot (1-(1-1/k)^\ell) \enspace. \qedhere
\]
\end{proof}


\end{document}

