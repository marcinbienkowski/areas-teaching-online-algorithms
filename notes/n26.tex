
\section{Szeregowanie zadań}

Choć szeregowanie zadań można rozpatrywać jako inny problem pakowania pojemników, przyjrzymy 
się definicji tego problemu w oryginalnym kontekście.

\problem{Szeregowanie zadań}{Danych jest $m$ identycznych procesorów. 
Sekwencja wejściowa składa się z zadań o określonym 
czasie wykonania. Należy przyporządkować dane zadanie do jednego z procesorów. Niech $\ell_i$ będzie sumą 
zadań przypisanych do procesora $i$. Chcemy, żeby $\max_i \{ \ell_i \}$ było jak najmniejsze.
}



\begin{theorem}
\label{lem:balance}
Algorytm $\GREEDY$ dla problemu szeregowania zadań na $m$ procesorach jest 
$(2 - 1/m)$-kon\-ku\-ren\-cyj\-ny
\end{theorem}

\begin{proof}
Weźmy dowolny ciąg wejściowy $\sigma$. Bez straty ogólności możemy założyć, że najbardziej 
obciążona będzie pierwszy procesor. Niech $w$ oznacza wagę ostatniego zadania na pierwszym procesorze, 
a $s$ sumaryczną wagę pozostałych zadań przypisanych do tego procesora. Obciążenie każdego innego procesora
musi również wynosić co najmniej $s$, bo inaczej $w$ zostałoby przypisane do innego procesora. 
Zatem sumaryczna waga wszystkich zadań to $w + s \cdot m$ i stąd
\[
	\OPT(\sigma) \geq \frac{w + s \cdot m}{m} = \frac{w}{m} + s \enspace.
\]
Jednocześnie $\OPT(\sigma) \geq w$. Zatem:
\begin{align*}
	\GREEDY(\sigma) \;=\; & w + s \\
		\leq\; & w + \OPT(\sigma) - \frac{w}{m} \\
		\leq\; & \OPT(\sigma) + w \cdot \left (1 - \frac{1}{m} \right) \\
		\leq\; & \OPT(\sigma) \cdot \left( 2 - \frac{1}{m} \right) \enspace. \qedhere
\end{align*}

\end{proof}

%%%%%%%%%%%%%%%%%%%%%%%%%%%%%%%%%%%%%

Rozważmy teraz zmodyfikowany problem szeregowania zadań. Zadanie $i$ w sekwencji wejściowej 
ma przypisaną wagę $w_i$ oraz zbiór procesorów $\P_i$. To zadanie można przyporządkowywać 
tylko do procesora ze zbioru $\P_i$.  

\begin{theorem}
\label{lem:balance}
Algorytm $\GREEDY$ dla problemu szeregowania ograniczonych zadań na $m$~procesorach jest 
$(\lceil \log m \rceil + 1)$-konkurencyjny
\end{theorem}

\begin{proof}
Mamy $m$ procesorów i wejście $\sigma$ składające się z $n$ zadań, gdzie $i$-te zadanie ma wagę $w_i$ i 
zestaw dopuszczalnych procesorów $\mathcal{P}_i$. 
Algorytm zachłanny szereguje je w sposób przedstawiony na rysunku \ref{fig:greedy_restricted}.
Zapis ,,$w = x \; (\to j)$'' oznacza, że jest to zadanie o wadze $x$ i zostało przyporządkowane 
w optymalnym rozwiązaniu do procesora $j$.

Dla potrzeb analizy dzielimy obciążenie na warstwy, każda z warstw ma grubość $\OPT(\sigma)$. 
Wystarczy, że pokażemy, że wielkość obciążenia każdego z procesorów mieści się w 
$\lceil \log n \rceil + 1$ warstwach.
W tym celu skoncentrujmy się na jednej warstwie (na rysunku jest to pierwsza warstwa);
będziemy ją nazywać {\em bieżącą warstwą}, zbiór warstw które po niej następują będziemy nazywać 
{\em następnymi warstwami}, a zbiór warstw je poprzedzających nazywamy {\em poprzednimi warstwami}.
Niech $W_j$ będzie obciążeniem procesora $j$ w bieżącej warstwie a $R_j$ sumarycznym obciążeniem
tego procesora w kolejnych warstwach. Oczywiście jeśli $W_j < \OPT(\sigma)$ to $R_j = 0$. 
Poniżej pokażemy, że 
\begin{equation}
\label{eq:exponential_decrease}
	\sum_{j = 1}^m W_j \;\geq\; \sum_{j = 1}^m R_j \enspace.
\end{equation}

\myfigure{greedy_restricted}{Przykładowe szeregowanie zadań przez algorytm \GREEDY}
	{\includegraphics[scale=0.8]{pict/sched}}

Aby udowodnić powyższą nierówność weźmy dowolny procesor $k$ (na rysunku $k = 1$). 
Pokażemy, że część $\sum_{j = 1}^m R_j$, która jest przeznaczona w rozwiązaniu optymalnym na procesor  
$k$ (jej wielkość oznaczamy przez $A_k$) jest mniejsza bądź równa $W_k$. 
Ponieważ z definicji te części te są rozłączne i pokrywają zawartość następnych warstw dostaniemy, że
$\sum_{k = 1}^m W_k \geq \sum_{k = 1}^m A_k = \sum_{j = 1}^m R_j$. 

Jak duże jest $A_k$? Niech $O_k$ oznacza zbiór wszystkich zadań przeznaczonych w optymalnym rozwiązaniu
na procesor $k$ (ciemniejsze zadania na rysunku). Oczywiście $A_k \leq \sum_{j \in O_k} w_j \leq \OPT(\sigma)$.
Jeśli $W_k = \OPT(\sigma)$, to 
nierówność $A_k \leq W_k$ jest trywialnie spełniona. Załóżmy więc, że $W_k < \OPT(\sigma)$. 

Aby obliczyć $A_k$, należy wziąć 
wartość $\sum_{j \in O_k} w_j$ i pomniejszyć ją o te zadania (lub ich części) z $O_k$, które leżą 
w bieżącej warstwie lub warstwach poprzednich. 
Zauważmy, że każde zadanie ze zbioru $O_k$ musi w rozwiązaniu produkowanym przez $\GREEDY$ 
zaczynać się wcześniej niż w chwili $W_k$. W przeciwnym przypadku procesor $k$ zostałby wykorzystany
dla obsługi tego zadania. Musimy od każdego zadania odjąć co najmniej $\OPT(\sigma) - W_k$, 
a zatem otrzymujemy
\[
	A_k \leq \sum_{j \in O_k} w_j - (\OPT(\sigma) - W_k) \leq W_k \enspace.
\]

Teraz pokażemy jak z nierówności \ref{eq:exponential_decrease} wynika
teza twierdzenia. Niech $L_i$ oznacza sumę wag zadań w warstwie $i$ i późniejszych. Wtedy 
$L_1 = \sum_{i=1}^n w_i$. Z (\ref{eq:exponential_decrease}) mamy, że 
\[ L_{i+1} \leq L_i / 2 \enspace, \]  
a stąd
\[ L_{\lceil \log m \rceil + 1} \leq \frac{L_1}{m} = \frac{\sum_{i=1}^n w_i}{m} \leq \OPT(\sigma) 
\enspace. \]
Oznacza to, że w 
warstwie $\lceil \log m \rceil + 1$ mamy tak mało wagi do rozdysponowania między procesory, że dowolne
przypisanie zadań (niekoniecznie zachłanne) gwarantuje zakończenie wykonywania zadań przed końcem tej warstwy.
\end{proof}

