\documentclass[a4paper,11pt]{scrartcl}
\usepackage[margin=2cm,bottom=2cm]{geometry}
\usepackage{amsmath,amsthm,amsopn,amsfonts}
\usepackage{polski}
\usepackage[utf8]{inputenc}
\usepackage{enumerate}
\usepackage{graphicx}
\pagestyle{empty}

\newcommand{\makeheader}[1]{
\begin{center}
{\Large\bf\sectfont Algorytmy online\\}
\vspace{0.3cm}
{\large\bf\sectfont Lista #1}\\
\end{center}
\vspace{0.7cm}
}


\newcommand{\myfigure}[3]{
\begin{figure}[t]
\centering
\fboxsep7pt
\framebox[0.95\textwidth]{
    \begin{minipage}{0.95\textwidth}
        \centering #3
    \end{minipage}
}
\caption{#2}
\label{fig:#1}
\end{figure}
}



\newcommand{\makefooter}{
\bigskip
\hfill {\em Marcin Bieńkowski}
}

\newcommand{\E}{\mathbf{E}}
\newcommand{\ALG}{\textsc{Alg}}
\newcommand{\OPT}{\textsc{Opt}}
\newcommand{\e}{\textnormal{e}}
\newcommand{\CNT}{\textsc{Cnt}}


\begin{document}
\makeheader{EXTRA}


\section{Zadania niewykorzystane}

\begin{description}

\item[Zadanie]
Pokaż, że \textsc{Fifo} jest nie jest algorytmem zaznaczającym. Pokaż, że \textsc{Fifo} 
jest $k$-konkurencyjny.

\item[Zadanie] 
Problem szukania krowy można też zdefiniować w następujący sposób. Mamy robota
umieszczonego początkowo w punkcie $(0,0)$ i prostą o równaniu $x = a$, gdzie
$a$ jest nieznane robotowi. Należy jak najszybciej znaleźć tę prostą (tj.
przejść przez nią w dowolny sposób). Oczywiście każdy rozsądny robot będzie
szukać jej tylko wzdłuż osi OX.

Rozważmy teraz modyfikację tego problemu, w której wiemy, że prosta ma
równanie $x = a$ albo $y = a$, tj.~wiemy, że jest równoległa do jednej z
osi układu współrzędnych, ale nie wiemy do której. Liczba $a$ jest
nieznana robotowi, ale można założyć, że jest liczbą całkowitą. Oczywiście
$\textsc{Opt} = a$.

Skonstruuj deterministyczny algorytm, który chodzi tylko po osiach OX lub
OY, którego współczynnik konkurencyjności wynosi co najwyżej $25$.

\item[Zadanie] 
Dla problemu z poprzedniego zadania, skonstruuj deterministyczny algorytm 
(niekoniecznie chodzący po osiach OX i OY) o współczynniku konkurencyjności co najwyżej $14$. 

\item[Zadanie]
    Pokaż, dolne ograniczenie $2+\frac{1}{2D}$ na konkurencyjność dowolnego randomizowanego algorytmu {\ALG}. 
    Wskazówka: załóż że {\ALG} jest zdefiniowany w każdym kroku jako rozkład prawdopodobieństwa. Na generowanym
    ciągu wejściowym uruchom jednocześnie {\ALG}, \textsc{EDGE} i {\OPT} starając się krzywdzić {\ALG} nie mniej niż 
    \textsc{EDGE} i jednocześnie gwarantować, że koszt \textsc{EDGE} jest odpowiednio wysoki w stosunku do {\OPT}.

\item[Zadanie] 
Rozważmy następujący algorytm \textsc{Cnt} dla problemu przenoszenia pliku w klice.
Z każdym wierzchołkiem $v$ \textsc{Cnt} związuje licznik $c_v$, początkowo równy $0$.
Algorytm ma dwa tryby działania: {\em tryb zwykły} i {\em tryb przeczekiwania}.
\begin{itemize}
\item W trybie zwykłym jeśli żądanie jest w wierzchołku $u$, $c_u$ jest zwiększany o $1$.
Następnie jeśli $c_u = D$, plik jest przenoszony do $u$, licznik $c_u$ jest ustawiany na $0$
i algorytm przechodzi do trybu przeczekiwania.
\item W trybie przeczekiwania algorytm odczekuje $D$ odwołań do wierzchołków innych niż
ten, który ma plik. Żadne liczniki nie są zmieniane podczas tego trybu.
\end{itemize}
Pokaż, że algorytm \textsc{Cnt} jest $3$-konkurencyjny.
Wskazówka: podziel całą sekwencję na fazy; jedna faza to $D$ odwołań do określonego
wierzchołka + kroki w trybie przeczekiwania po przeniesieniu pliku do tego wierzchołka.
Zauważ, że każdy krok należy do dokładnie jednej fazy.

% koszt v w fazie zwiazanej z v = koszt requestow w krokach nalezacych do v 
% + koszt przenosin Z v DO czegos innego w I, gdzie I jest minimalnym przedzialem czasowym 
% zawierajacym wszystkie kroki z tej fazy.

\item[Zadanie]
Pokaż, że konkurencyjność dowolnego deterministycznego algorytmu dla problemu 
szeregowania zadań na $m$ niepowiązanych maszynach wynosi co najmniej $\Omega(\log m)$.

\item[Zadanie]
Pokaż, że dla każdego $\gamma$-konkurencyjnego algorytmu dla problemu szeregowania zadań
i dla dowolnie małego $\epsilon > 0$, istnieje algorytm który jest ściśle 
$(\gamma+\epsilon)$-konkurencyjny.
% zadania mozna skalowac

\item[Zadanie] Pokaż, że na podstawie algorytmu $EXP_\lambda$ dla problemu minimalizacji
	obciążenia (z wykładu), można skonstruować algorytm, który będzie $O(\log m)$ konkurencyjny. 
	Wskazówka: Kiedy koszt $\OPT$ przekracza $\lambda$, podwajaj $\lambda$.
	% jesli algorytm dojedzie do lambda = 2^k, to zaplaci sumarycznie 2^{k+1} * log m, a OPT placi 
	% co najmniej 2^{k-1} (bo faza k-1 sie zakonczyla, wiec szacowanie OPT <= 2^{k-1} nie bylo prawdziwe) 

\item[Zadanie]
	Pokaż, że w następującym grafie skierowanym $G$, dolne ograniczenie na (ścisłą) konkurencyjność 
	deterministycznego algorytmu dla problemu minimalizacji
	obciążenia wynosi $\Omega(\log m)$. $G$ składa się z $\binom{k}{2}$ wierzchołków źródłowych $s_{i,j}$, 
	$1 \leq i \neq j \leq k$, $k$ wierzchołków pośrednich $u_i$, $1 \leq i \leq k$ i jednego ujścia $t$.
	Dowolny wierzchołek źródłowy $s_{i,j}$ jest połączony z $u_i$ i $u_j$, a dowolny wierzchołek
	pośredni $u_i$ jest połączony z ujściem $t$. W takim grafie $m = 2 \cdot \binom{k}{2} + k$.
	Wskazówka: idea rozwiązania jest dokładnie taka sama jak w przypadku dolnego ograniczenia na problem 
	szeregowania ograniczonych zadań.

\item[Zadanie]
	Rozważmy następujący uogólniony problem szeregowania zadań na $m$ procesorach: z zadaniem $t$
	jest związany wektor $m$ liczb $(b_t(1), b_t(2), b_t(3), \ldots, b_t(m))$, gdzie $b_t(i)$ jest 
	kosztem przyporządkowania do procesora $i$ (czasem wykonania na procesorze $i$).
	Rozważmy następujący algorytm $A_\lambda$, który do wykonania zadania $t$ wybiera procesor $i$
	minimalizujący
	 \[	a^\frac{L_{t-1}(i) + b_t(i)}{\lambda} - a^\frac{L_{t-1}(i)}{\lambda} \enspace, \]
	gdzie $a = 3/2$ a $L_{t-1}(i)$ oznacza obciążenie procesora $i$ na początku kroku $t$.

	Pokaż, że na sekwencji, której rozwiązanie optymalne ma koszt co najwyżej $\lambda$, 
	koszt $A_\lambda$ wynosi $O(\log m) \cdot \lambda$.
	Jak na tej podstawie skonstruować algorytm $O(\log m)$-konkurencyjny?
	% dowód jest taki sam jak w przypadku routingu, tylko graf ma dwa wierzcholki 
	% polaczone m krawedziami.

\item[Zadanie]
Rozważ modyfikację algorytmu dla ułamkowego problemu pokrywania zbiorami podanego na wykładzie.
Mianowicie podczas rozważania elementu $e_j$ w kroku $j$ przypisanie $x_S \leftarrow (1+1/c_S) \cdot x_S + 1/(m \cdot c_S)$ 
zastępujemy przypisaniem 
\[
    x_S \leftarrow \left(1+\frac{1}{c_S}\right) \cdot x_S + \frac{1}{|\{S: e_j \in S\}| \cdot c_S}
\]
zaś $y_j$ zwiększany jest po prostu o $1$. 
Pokaż, że (ścisłą) konkurencyjność algorytmu zatrzymanego po $k$ krokach 
można ograniczyć przez $O(\log d)$, gdzie $d = \max_{1 \leq j \leq k} |\{S: e_j \in S\}|$. 
Uwaga: generowane przez algorytm
rozwiązania programu dualnego będą dopuszczalnymi rozwiązaniami, dopiero gdy 
podzielimy je przez $\Omega(\log d)$.


\item[Zadanie]
Rozważmy następujący proces losowego zaokrąglania rozwiązania ułamkowego $\{ 
x_S \}_{S \in \mathcal{F}}$ dla problemu pokrywania zbiorami.  Dla każdego
zbioru $S$ w kroku $k$ losujemy liczby $T_{S,j}$ z jednostajnym rozkładem z
odcinka $[0,1]$, gdzie $j \in \{1,2,\ldots, \ell \cdot \lceil \ln (k+1)
\rceil\}$ a $\ell$ jest pewną stałą.\footnote{Łatwo osiągnąć monotoniczność
generowanego rozwiązania (tj.~własność, że algorytm nie usuwa z rozwiązania już 
przyjętych zbiorów): w każdym kroku dolosowujemy brakujące liczby, ale nie 
zmieniamy już istniejących.} Algorytm randomizowany bierze zbiór $S$ do
rozwiązania jeśli $x_S \geq \min_j \{T_{S,j}\}$.



\item[Zadanie]
Rozważamy problem Adwords w wariancie gdzie
budżet każdego gracza (reklamodawcy) wynosi $b = 1$, gracz jest zainteresowany
pewnymi słowami kluczowymi i jest za nie skłonny zapłacić $1$, a za pozostałe
nie chce płacić w ogóle. Innymi słowy: każdego gracza stać na jednokrotne
wyświetlenie jego reklamy.

Pokaż, że dla dowolnego $n$ i dowolnego algorytmu deterministycznego
\DET można stworzyć taką instancję wejściową, że \OPT zyskuje na
niej $n$ zaś \DET zyskuje co najwyżej $\lceil n / 2 \rceil$.

\item[Zadanie] 
Rozważamy problem Adwords w wariancie gdzie
budżet każdego gracza (reklamodawcy) wynosi $b = 1$, gracz jest zainteresowany
pewnymi słowami kluczowymi i jest za nie skłonny zapłacić $1$, a za pozostałe
nie chce płacić w ogóle.

Rozważmy algorytm \textsc{Random}, który dla danego słowa kluczowego wybiera losowego gracza, który 
ma jeszcze pieniądze i jest zainteresowany takim słowem kluczowym. Skonstruuj instancję, taką 
że \OPT zyskuje na niej $n$, zaś oczekiwany zysk \textsc{Random} wynosi co najwyżej 
$n/2 + o(n)$.\footnote{Istnieją instancje dla których zysk \textsc{Random} to $n/2 + O(\log n)$.}

\item[Zadanie]
Pokaż dolne ograniczenie w wysokości $\Omega (\log N)$ na ścisłą
konkurencyjność randomizowanego algorytmu routingu na linii 
zawierającej $N+1$ wierzchołków. 

Wskazówka: skorzystaj z zasady minimaksowej i
rozważ ciąg $[0,N], [0,N/2], [N/2,N], [0,N/4],$ $[N/4, N/2]$, $[N/2,3/4 N], [3/4
N, N], \ldots,$
$[0,1],[1,2],[2,3],[3,4],\ldots, [N-2,N-1], [N-1,N]$; z~pewnymi
prawdopodobieństwami na wejściu pojawia się tylko prefiks tego ciągu.


\item[Zadanie]
Pokaż, że współczynnik (ścisłej) konkurencyjności dowolnego algorytmu dla problemu szeregowania 
ograniczonych zadań na $m$ procesorach wynosi co najmniej $\Omega(\log m)$. 
	%Niech $m = 2^k$. Weźmy ciąg wejściowy $\sigma$ zawierający $m$ zadań. Zadania (wszystkie o wadze 
%$1$) będą grupowane w partie, $i$-ta partia zawiera $m \cdot 2^{-i}$ zadañ. Każde z zadań będzie 
%mogło być wykonywane na dwóch maszynach. Pokaż, że można tak dobrać te dwie maszyny dla każdego 
%zadania, że $\OPT(\sigma) = 1$ a $\ALG(\sigma) \geq \lceil \log (m+1) \rceil$.



\item[Zadanie]
Weźmy dowolny graf w którym minimalna odległość między dwoma różnymi punktami
wynosi $m$ a maksymalna $M$.  Pokaż, że jeśli istnieje $k$-konkurencyjny
algorytm dla problemu pamięci podręcznej, to istnieje $(k \cdot
M/m)$-konkurencyjny algorytm dla problemu $k$ serwisantów w tym grafie. 


\end{description}

\makefooter
\end{document}



