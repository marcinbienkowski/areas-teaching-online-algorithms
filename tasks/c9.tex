\documentclass[a4paper,11pt]{article}
\usepackage[margin=2cm,bottom=2cm]{geometry}
\usepackage{amsmath,amsthm,amsopn,amsfonts}
\usepackage{polski}
\usepackage[utf8]{inputenc}
\usepackage{enumerate}
\usepackage{graphicx}
\pagestyle{empty}

\newcommand{\makeheader}[1]{
\begin{center}
{\Large\bf\sectfont Algorytmy online\\}
\vspace{0.3cm}
{\large\bf\sectfont Lista #1}\\
\end{center}
\vspace{0.7cm}
}


\newcommand{\myfigure}[3]{
\begin{figure}[t]
\centering
\fboxsep7pt
\framebox[0.95\textwidth]{
    \begin{minipage}{0.95\textwidth}
        \centering #3
    \end{minipage}
}
\caption{#2}
\label{fig:#1}
\end{figure}
}



\newcommand{\makefooter}{
\bigskip
\hfill {\em Marcin Bieńkowski}
}

\newcommand{\E}{\mathbf{E}}
\newcommand{\ALG}{\textsc{Alg}}
\newcommand{\OPT}{\textsc{Opt}}
\newcommand{\e}{\textnormal{e}}
\newcommand{\CNT}{\textsc{Cnt}}


\begin{document}
\makeheader{9}


\begin{description}

\item[Zadanie 1.] 
Chcemy za pomocą jednej transakcji zamienić jedngo bitcoina na złotówki w ciągu najbliższych $T$ dni. 
Każdego dnia dowiadujemy się, jaki jest kurs danego dnia; jest on 
dowolną liczbą rzeczywistą z zakresu $[1,M]$. Wartości $T$ i $M$ są znane algorytmowi od początku.
\begin{enumerate}
\item \textbf{(1 pkt.)} Skonstruuj deterministyczny algorytm, który będzie  
    ściśle $\sqrt{M}$-konkurencyjny.
\item \textbf{(2 pkt.)} Skonstruuj randomizowany algorytm, który będzie  
    ściśle $O(\log M)$-konkurencyjny.
\end{enumerate}


%\item[Zadanie 1] \textbf{(2 pkt.)}
%Dla dowolnych dwóch punktów $x$ i $y$ na prostej przez $x \preceq y$ oznaczamy, że $x$
%leży na lewo od $y$ lub w tym samym miejscu co $y$.  Ustalmy dowolną pozycję
%punktów $a_1 \preceq a_2 \preceq \ldots \preceq a_k$ i $o_1 \preceq o_2 \preceq
%\ldots \preceq o_k$. Załóżmy, że dla pewnych $i$ oraz $j$
%zachodzi $a_j \preceq o_i \preceq a_{j+1}$. Pokaż, że istnieje doskonałe 
%skojarzenie o minimalnej 
%wadze między zbiorami $\{a_i\}_i$ i $\{o_i\}_i$, takie, że $o_i$ jest skojarzony albo 
%z $a_j$ albo z $a_{j+1}$.


\item[Zadanie 2.] \textbf{(2 pkt.)}
Rozważmy problem $k$ serwisantów na okręgu, tj.~przestrzeń, w której występują
odwołania jest okręgiem, a~odległość między dwoma punktami okręgu
jest najkrótszą odległością liczoną wzdłuż
okręgu. Rozważmy następujący randomizowany algorytm $\CIRC$:
\begin{quote}
Na początku sekwencji wejściowej wybierz losowo jeden punkt $P$ z
okręgu. $P$ jest barierą, która dzieli okrąg na odcinek  (o
początku i końcu w punkcie $P$). Następnie do obsługi żądań $\CIRC$ uruchamia
algorytm \textsc{Double Coverage} na tym odcinku (tj.~nigdy nie przekracza punktu
$P$). 
\end{quote}
Pokaż, że $\CIRC$ jest $2 k$-konkurencyjny.


\item[Zadanie 3.] \textbf{(3 pkt.)}
Rozważmy następujący algorytm rozwiązujący problem 2 serwisantów na płaszczyźnie 
euklidesowej. Aby obsłużyć żądanie w punkcie $r$, algorytm wykonuje następujące kroki:
\begin{enumerate}
\item Niech $x$ będzie serwisantem bliższym $r$ zaś $y$ dalszym (remisy rozstrzygamy 
dowolnie).
\item Przesuń $y$ o $\frac{1}{2} \cdot (d(x,r) + d(x,y) - d(y,r))$ 
w stronę $x$.\footnote{Na prostej algorytm zachowuje się jak \textsc{Double Coverage}.}
\item Przesuń $x$ do $r$. 
\end{enumerate}
Pokaż że powyższy algorytm jest $O(1)$-konkurencyjny. \emph{Wskazówka:} wykorzystaj funkcję 
potencjału podobną do tej wykorzystywanej w algorytmie \textsc{Double Coverage}. 


\item[Zadanie 4.] \textbf{(2 pkt.)}
Rozważmy graf czterowierzchołkowy będący kwadratem o równych długościach
krawędzi.  Rozważmy następujący algorytm randomizowany $\ALG$ dla problemu
dwóch serwisantów w tym grafie.  $\ALG$ rozpoczyna z serwisantami w dwóch różnych
wierzchołkach. Aby obsłużyć żądanie w punkcie $r$:
\begin{enumerate}
\item jeśli $\ALG$ ma serwisanta w $r$, to nic nie robi;
\item w przeciwnym przypadku do $r$ przesuwany jest ten z serwisantów, który jest bliższy $r$;
\item jeśli obaj serwisanci mają taką samą odległość do $r$, to serwisant do
przesunięcia wybierany jest przez rzut monetą.
\end{enumerate}
Pokaż, że $\ALG$ jest $O(1)$-konkurencyjny. 
Jakie jest najlepsze ograniczenie jakie potrafisz podać?


\end{description}

\makefooter
\end{document}

