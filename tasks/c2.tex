\documentclass[a4paper,11pt]{article}
\usepackage[margin=2cm,bottom=2cm]{geometry}
\usepackage{amsmath,amsthm,amsopn,amsfonts}
\usepackage{polski}
\usepackage[utf8]{inputenc}
\usepackage{enumerate}
\usepackage{graphicx}
\pagestyle{empty}

\newcommand{\makeheader}[1]{
\begin{center}
{\Large\bf\sectfont Algorytmy online\\}
\vspace{0.3cm}
{\large\bf\sectfont Lista #1}\\
\end{center}
\vspace{0.7cm}
}


\newcommand{\myfigure}[3]{
\begin{figure}[t]
\centering
\fboxsep7pt
\framebox[0.95\textwidth]{
    \begin{minipage}{0.95\textwidth}
        \centering #3
    \end{minipage}
}
\caption{#2}
\label{fig:#1}
\end{figure}
}



\newcommand{\makefooter}{
\bigskip
\hfill {\em Marcin Bieńkowski}
}

\newcommand{\E}{\mathbf{E}}
\newcommand{\ALG}{\textsc{Alg}}
\newcommand{\OPT}{\textsc{Opt}}
\newcommand{\e}{\textnormal{e}}
\newcommand{\CNT}{\textsc{Cnt}}


\begin{document}
\makeheader{2}


\begin{description}

\item[Zadanie 1.] \textbf{(2 pkt.)} 
Aby zrobić miejsce w pamięci podręcznej, algorytm \FIFO usuwa z niej stronę,
która była tam najdłużej. Pokaż, że \FIFO jest nie jest
algorytmem zaznaczającym. Pokaż, że \FIFO jest $k$-konkurencyjny.

\item[Zadanie 2.] \textbf{(2 pkt.)} 
{\LFU} (\emph{Least Frequently Used}) to algorytm, który z każdą stroną wiąże
licznik określający, ile odwołań wystąpiło do tej strony. Zapisanie strony w
pamięci podręcznej lub wyrzucenie jej stamtąd nie powoduje zerowania takiego
licznika. Aby zrobić miejsce w~pamięci podręcznej, {\LFU} wyrzuca z niej stronę
o~najmniejszej wartości licznika. Pokaż, że {\LFU} nie jest konkurencyjny. 


\item[Zadanie 3.] \textbf{(2 pkt.)} 
Celem tego zadania jest udowodnienie, że konkurencyjność dowolnego algorytmu deterministycznego \DET dla problemu pamięci podręcznej wynosi co najmniej $k$. 
Założymy, że algorytm \DET startuje z pełną pamięcią podręczną. 
Rozważ ciąg wejściowy $\sigma^n$ o długości~$n$ składający się z $k+1$ różnych stron, w którym adwersarz zawsze pyta o stronę, której \DET nie ma w pamięci podręcznej.

\begin{enumerate}
\item Pokaż, że $\OPT(\sigma^n) \leq 1 + n/k$. \emph{Wskazówka: jak na ciągu $\sigma^n$ zachowa się algorytm \emph{Longest Forward Distance}?}

\item Wywnioskuj, że $\DET(\sigma^n) \geq k \cdot \OPT(\sigma^n) - k$.

\item Jak z poprzedniego punktu wynika dolne ograniczenie na konkurencyjność \DET? 
\end{enumerate}


\item[Zadanie 4.] 
Załóżmy, że algorytm dysponuje pamięcią podręczną o rozmiarze $k$,
a optymalny algorytm pamięcią o rozmiarze $h \leq k$.
\begin{enumerate}
\item \textbf{(1 pkt.)} Pokaż, że dowolny deterministyczny algorytm zaznaczający jest $\frac{k}{k-h+1}$-kon\-ku\-ren\-cyj\-ny.
\item \textbf{(3 pkt.)} Pokaż, że konkurencyjność dowolnego algorytmu deterministycznego (niekoniecznie zaznaczającego) wynosi 
co najmniej $\frac{k}{k-h+1}$.
\end{enumerate}


%\item[Zadanie.] Rozważmy randomizowany algorytm zaznaczający {\RMARK} dla problemu pamięci %podręcznej.
%\begin{enumerate}
%\item \textbf{(1 pkt.)} Pokaż, że jeśli pamięć RAM zawiera $n = k+1$ stron, to algorytm 
%{\RMARK} jest $H_k$-konkurencyjny. 
%\item \textbf{(3 pkt.)} Pokaż, że dla dowolnego $k \geq 2$ można dobrać takie $n$, że %konkurencyjność {\RMARK} jest większa niż $H_k$.
%\end{enumerate}

\end{description}

\makefooter
\end{document}

