
%%%%%%%%%%%%%%%%%%%%%%%%%%%%%%%%%%%%%%%
% Theorem-style environments
%%%%%%%%%%%%%%%%%%%%%%%%%%%%%%%%%%%%%%%

\newtheorem{theorem}{Twierdzenie}[section]
\newtheorem{lemma}[theorem]{Lemat}
\newtheorem{corollary}[theorem]{Wniosek}
\newtheorem{definition}[theorem]{Definicja}
\newtheorem{note}[theorem]{Uwaga}
\newtheorem{observation}[theorem]{Obserwacja}
\newtheorem{exercise}[theorem]{Ćwiczenie}
\newtheorem{example}[theorem]{Przykład}

%\newtheoremstyle{note}{6pt}{6pt}{\itshape}{}{\itshape}{.}{.5em}{}
%\theoremstyle{note}
%\newtheorem{observation}[theorem]{Obserwacja}
%\theoremstyle{plain}

\def\proofname{Dowód}
\renewcommand{\openbox}{\leavevmode\rule{1.4ex}{1.4ex}}
\makeatletter
 \renewenvironment{proof}[1][\proofname]{\par
 \pushQED{\qed}%
 \normalfont \topsep6\p@\@plus6\p@\relax
 \trivlist
 \item[\hskip\labelsep
 \bfseries
 #1\@addpunct{.}]\ignorespaces
 }{%
 \popQED\endtrivlist\@endpefalse
 }
\makeatother

\newcommand{\problem}[2]{
\vspace{0.15cm}
\noindent
\fboxsep7pt
\framebox[\textwidth]{
\begin{minipage}{0.95\textwidth}
\textsc{#1}. #2
\end{minipage}
}
\medskip
}


\newcommand{\myfigure}[3]{
\begin{figure}[t]
\centering
\fboxsep7pt
\framebox[0.95\textwidth]{
	\begin{minipage}{0.95\textwidth}
		\centering #3
	\end{minipage}
}
\caption{#2}
\label{fig:#1}
\end{figure}
}

\newcommand{\myfigurehere}[2]{
\begin{figure}[h]
\centering
\fboxsep7pt
\framebox[0.95\textwidth]{
	\begin{minipage}{0.95\textwidth}
		\centering #2
	\end{minipage}
}
\label{fig:#1}
\end{figure}
}


\newcommand{\rozdzial}[1]{
\input n#1
\vspace{0.3cm}
\newpage
}

\newcommand{\etalchar}[1]{$^{#1}$}

\newcommand{\myparagraph}[1]{%
\medskip
\noindent
{\bf #1.}
}

\input shortcut

